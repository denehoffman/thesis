The $K_S^0K_S^0$ channel at GlueX potentially contains one of the missing pieces of QCD predicted by lattice simulations, the scalar glueball. However, as we have seen, a direct search for glueballs in this channel is impeded by the presence of multiple overlapping resonant states in its expected mass range which are indistinguishable from it in every physical observable to which we have access. With this in mind, there are still interesting observations to be made, and further studies using these analysis techniques may someday lead to a glueball discovery.

We have shown that the background in this channel can be efficiently reduced using an sPlot weighting technique on the rest-frame lifetime of the kaons. This approach has allowed us to isolate kaon candidates without a reliance on the kaon mass, reducing effects from pion production without decaying kaon states. Instead, we kinematically constrain the masses of each kaon, which, along with a cut on the $\chi^2_\nu$ of the kinematic fit, yields a spectrum of meson candidates with a high mass resolution.

To model the data in this channel, we began with a standard formalism for spin-helicity states and extended it to linearly polarized photoproduction amplitudes. This mass-independent model can tell use the distribution of spin-$0$ and spin-$2$ resonances in our data, and we find a significant enhancement in the $D_2$ projection which is consistent with the expected spin-$2$ resonances in the system and contains a large contribution from the $f_2'(1525)$ resonance. Breaking the $S$-wave into reflectivity components, we find that most of the spectrum is dominated by the positive reflectivity component except for a sharp peak around $\SI{1.5}{\giga\electronvolt}$, which may be indicative of a different production mechanism for the $f_0(1500)$.

While recognizing that we could never truly separate the resonances in this channel by isospin, we nonetheless attempted to model the mass distribution using a fixed $K$-matrix distribution. This approach led to a model with numerous local minima, which we attempted to circumvent using a guided method. While we have shown that this method yields results which mostly agree with the mass-independent model, the underlying resonances are not well-constrained, and the uncertainties on overlapping resonances which differ by isospin only, like the $a_0(980)$ and $f_0(980)$, are too large to draw any conclusions about their relative contributions.

While this part of the study was inconclusive, it may still be useful if a coupled-channel analysis of GlueX data is ever performed. Combining this channel with isospin-$0$-exclusive channels like $\pi\pi$ and isospin-$1$-exclusive channels like $\eta(')\pi$ may someday yield a more complete picture of the light flavorless meson spectrum. While the eventual goal would be to use a $K$-matrix with multiple GlueX channels, floating decay couplings, and pole masses, the intermediate use of a fixed $K$-matrix is an important study and validation step in this process. For example, the $K_S^0K_S^0$ channel may help constrain known resonances in these channels, raising the statistical significance of potential hybrid mesons which may be observable at GlueX, and likewise, these channels, which do not couple to the glueball, may also help to constrain the myriad resonances in $K_S^0K_S^0$. Additional constraints may come from and benefit channels like $K^+K^-$ and its overlap with $K_S^0K_L^0$, both of which are also being studied at GlueX. Further analysis techniques may include alternate methods of global optimization, such as stochastic methods like those used in many machine learning problems.

This analysis has included all of the data available at time of writing, and while this constitutes the largest photoproduction dataset (and one of the largest datasets for this channel with any production mechanism) to date, the GlueX experiment is still ongoing with more data expected in the near future. The author hopes that this work may serve as a small stepping stone on the path towards a more complete understanding of the light meson spectrum, hybrids, exotics, and glueballs.
