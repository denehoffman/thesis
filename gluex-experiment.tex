GlueX, short for the Gluonic Excitation experiment located at the Thomas Jefferson National Accelerator Facility (JLab), first began collecting publication-ready data in 2016 with the goal of establishing the spectrum of light exotic states such as hybrid mesons and glueballs.

GlueX recieves an unpolarized electron beam from the Continuous Electron Beam Accelerator Facility (CEBAF), which is then converted to linearly polarized photons via coherent bremsstrahlung from a diamond radiator. The scattering electrons are detected in an array of high-resolution scintillators called the Tagger Microscope (TAGM) which covers beam energies between $8$ and $\SI{9}{\giga\eV}$, a region of energy refered to as the ``coherent peak''. This range of beam energies has been specifically tuned (by modification of the radiator properties) to provide the highest amount of polarization{\color{red}[FIGURE]}. The rest of the energy range, regions from about $3-\SI{8}{\giga\eV}$ and $9-\SI{12}{\giga\eV}$, are covered by the lower-resolution Tagger Hodoscope (TAGM). These taggers are used to determine the photon energy, since it is equal to the difference between the incident and outgoing electron energies~\cite{adhikari_gluex_2021}.

The photon beam next passes through a thin beryllium foil which induces $e^+e^-$ pair production. The azimuthal angle of the produced electrons is measured by the Triplet Polarimeter (TPOL) to measure the photon polarization. The Pair Spectrometer (PS) then counts these electrons in coincidence with the TAGM and TAGH detectors. A known fraction of the beam is converted to $e^+e^-$ pairs in this way, allowing for an accurate measurement of the photon flux and the energy dependence of the polarization fraction.

Next, the photon interacts with a liquid hydrogen cryotarget, which maintains a temperature of $\SI{20.1}{\K}$, allowing the contents to act as a stationary proton target. A set of thin scintillators called the Start Counter (ST) surrounds the target, which captures the initial signals (and azimuthal angles) from reaction products and associates them with the electron radio frequency (RF) beam bunch from which the reaction originated. Any products of the reaction next pass through either the Central Drift Chamber (CDC) or Forward Drift Chamber (FDC) depending on their trajectory (particles moving along the direction of the beamline pass through the FDC). These chambers are filled with wires which sense the fields generated by charged particles as they pass nearby, allowing for the reconstruction of charged trajectories. The entire detector is situated inside a solenoid with a magnetic field around $\SI{2}{\tesla}$, and this field bends the trajectories of charged particles, allowing for proper identification of the sign of the charge as well as the particle's momentum. These chambers also allow for reconstruction of decay vertices by tracing back charged trajectories to their nearest approach.

The Barrel Calorimeter (BCAL) surrounding the CDC and the Forward Calorimeter (FCAL) in front of the FDC measure the energy of photon showers from interactions with charged and neutral particles. This includes photons originating from the decays of light neutral mesons like the $\pi^0$ and $\eta$. Because such neutral particles pass through the CDC and FDC without detection, these calorimeters are necessary for their reconstruction. However, since only three vertices exist in such reconstructions (the initial vertex in the target and the final vertices in the calorimeter), GlueX is unable to reconstruct the decay vertices of these neutral particles. Since we are studying the a reaction with a fully charged final state, this might not seem important. However, in {\color{red}Section ?}, we will discuss the study of neutral decays of $K_S^0\to 2\pi^0 \to 4\gamma$, where the reconstruction will be severely limited due to the inability to reconstruct the decay vertex of the $K_S^0$.

There is a final scinatillator called the Time-of-Flight (TOF) detector situated immediately between the FDC and the FCAL. In combination with the ST, this aids in charged-particle identification via a measurement of the flight time between the two detectors. Additionally, in 2019, an additional detector, the DIRC\footnote{An acronym for Detection of Internally Reflected Cherenkov light} detector was installed between the TOF and FDC to further aid in identification of charged pions and kaons. This detector measures the angle of the cone of Cherenkov radiation emitted by relativistic charged particles, which can be used to ascertain their velocity via the relation $\cos\theta_c \sim 1/v$. When compared to measurements of the particle's momentum, this detector can be used to distinguish pions, kaons, and protons. Further information about the GlueX detector and the DIRC can be found in references \cite{adhikari_gluex_2021} and \cite{ali_installation_2020} respectively.

The data analyzed in this thesis were collected in four separate ``run periods'' across two experiment ``phases'', denoted Phase-I and Phase-II. Phase-I consists of three run periods, notated Spring 2017, Spring 2018, and Fall 2018 by the season and year when data collection began, and Phase-II contains one run period, Spring 2020\footnote{This run technically began at the end of 2019} and is the only dataset which was collected after the DIRC installation. The total luminosity, coherent peak range, and luminosity in the coherent peak for each run period is listed in \Cref{tab:run-info}.


\begin{table}
  \begin{center}
    \begin{tabular}{ccccc}\toprule
      Experiment & Run Period & Luminosity ($E_\gamma > \SI{6.0}{\giga\eV}$) & Coherent Peak Range & Luminosity in Coherent Peak \\\midrule
      Phase-I & Spring 2017 & $\SI{74.7}{\pico\barn^{-1}}$ & $8.2-\SI{8.8}{\giga\eV}$ & $\SI{21.8}{\pico\barn^{-1}}$ \\
              & Spring 2018 & $\SI{223.8}{\pico\barn^{-1}}$ & $8.2-\SI{8.8}{\giga\eV}$ & $\SI{63.0}{\pico\barn^{-1}}$ \\
              & Fall 2018 & $\SI{141.1}{\pico\barn^{-1}}$ & $8.2-\SI{8.8}{\giga\eV}$ & $\SI{40.1}{\pico\barn^{-1}}$ \\\midrule
      Phase-II & Spring 2020 & $\SI{386.2}{\pico\barn^{-1}}$ & $8.0-\SI{8.6}{\giga\eV}$ & $\SI{132.4}{\pico\barn^{-1}}$ \\\midrule
      Total & & $\SI{825.8}{\pico\barn^{-1}}$ & & $\SI{257.3}{\pico\barn^{-1}}$ \\\bottomrule
    \end{tabular}
    \caption{}\label{tab:run-info}
  \end{center}
\end{table}

The entire GlueX detector has been simulated with Geant4. Due to small changes and updates to the detector and its simulation between run periods, we will treat these datasets separately during the analysis, only presenting combined results in summary. Therefore, when generating Monte Carlo simulated data to model detector acceptance, each run period must be simulated separately.
