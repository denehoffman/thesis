GlueX, short for the Gluonic Excitation experiment located at the Thomas Jefferson National Accelerator Facility (JLab), first began collecting publication-ready data in 2016 with the goal of establishing the spectrum of light exotic states such as hybrid mesons and glueballs.

GlueX recieves an unpolarized electron beam from the Continuous Electron Beam Accelerator Facility (CEBAF), which is then converted to linearly polarized photons via coherent bremsstrahlung from a diamond radiator. The scattering electrons are detected in an array of high-resolution scintillators called the tagger microscope (TAGM) which covers beam energies between $8$ and $\SI{9}{\giga\eV}$, a region of energy refered to as the ``coherent peak''. This range of beam energies has been specifically tuned (by modification of the radiator properties) to provide the highest amount of polarization{\color{red}[FIGURE]}. The rest of the energy range, regions from about $3-\SI{8}{\giga\eV}$ and $9-\SI{12}{\giga\eV}$, are covered by the lower-resolution tagger hodoscope (TAGM). These taggers are used to determine the photon energy, since it is equal to the difference between the incident and outgoing electron energies~\cite{adhikari_gluex_2021}.

The photon beam next passes through a thin beryllium foil which induces $e^+e^-$ pair production. The azimuthal angle of the produced electrons is measured by the triplet polarimeter (TPOL) to measure the photon polarization. The pair spectrometer (PS) then counts these electrons in coincidence with the TAGM and TAGH detectors. A known fraction of the beam is converted to $e^+e^-$ pairs in this way, allowing for an accurate measurement of the photon flux and the energy dependence of the polarization fraction.

Next, the photon interacts with a liquid hydrogen cryotarget, which maintains a temperature of $\SI{20.1}{\K}$, allowing the contents to act as a stationary proton target.


The data analyzed in this thesis was collected in four separate ``run periods'' across two experiment ``phases'', denoted Phase I and Phase II. 
