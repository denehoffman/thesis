\section{A Brief History of Particle Physics}

Since the days of the ancient Greeks, scientists and philosophers alike have been interested in the fundamental question concerning the composition of the universe. While the Greeks maintained that the world was composed of four indivisible elemental substances (fire, earth, air, and water)~\cite{aristotle_metaphysics_350bce}, this was at best a guess by the early philosophers, who had no mechanism with which to test their theory. Ironically, these philosophers struggled with a question to which us modern physicists still have no answer: Are the building blocks of the natural world fundamental (indivisible)~\cite{aristotle_physics_350bce}?

In 1808, John Dalton published a manuscript which described what is now called the "law of multiple proportions" after compiling several observations on chemical reactions which occur with specific proportions of their reactants. He anglicized the Greek \textit{atomos}, meaning ``not able to be cut'', into the word we are familiar with\textemdash ``atom''~\cite{dalton_new_1808}.
