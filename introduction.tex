\section{A Brief History of Particle Physics}\label{sec:a_brief_history_of_particle_physics}

Since the days of the ancient Greeks, scientists and philosophers alike have been interested in the fundamental question concerning the composition of the universe. While the Greeks maintained that the world was composed of four indivisible elemental substances (fire, earth, air, and water)~\cite{aristotle_metaphysics_350bce}, this was at best a guess by the early philosophers, who had no mechanism with which to test their theory. Ironically, these philosophers struggled with a question to which us modern physicists still have no answer: Are the building blocks of the natural world fundamental (indivisible)~\cite{aristotle_physics_350bce}?

In 1808, John Dalton published a manuscript which described what is now called the "law of multiple proportions" after compiling several observations on chemical reactions which occur with specific proportions of their reactants. He anglicized the Greek \textit{atomos}, meaning ``not able to be cut'', into the word we are familiar with\textemdash ``atom''~\cite{dalton_new_1808}. Towards the end of the century, J. J. Thomson demonstrated that cathode rays could be deflected by an electrostatic field, an observation which could not be explained by the prevailing theory that the rays were some form of light~\cite{thomson_cathode_1897}. Instead, he proposed that these rays were made up of charged particles he called ``corpuscles'' (later renamed to the familiar ``electrons'')~\cite{thomson_corpuscular_1907}. Around the same time (between 1906 and 1913), Ernest Rutherford, Hans Geiger, and Ernest Marsden conducted experiments in which they scattered alpha particles through a thin metal foil, and, through an analysis of the scattering angles, concluded that a positively charged nucleus must exist at the center of atoms, surrounded by electrons~\cite{rutherford_lxxix_1911}.

Over the next several decades, the nucleus was further divided into protons and neutrons\footnote{For the discovery of the electron and neutron, Thomson and James Chadwick won Nobel Prizes in Physics in 1906 and 1935, respectively. Rutherford won the 1908 Nobel Prize in Chemistry for his research in radiation. However, I want to emphasize that while I mention the ``big names'' here, there are many who contributed in relative obscurity.}~\cite{masson_xxiv_1921,chadwick_possible_1932}. In 1964, Murray Gell-Mann and George Zweig proposed a theory that protons and neutrons (and all other baryons and mesons) were in fact composed of smaller particles Gell-Mann called ``quarks''\footnote{Upon reading the section of Finnegan's Wake which Gell-Mann cites as inspiration behind the name, I found (somewhat surprisingly) that the word ``quark'' was originally intended to rhyme with ``mark'', ``ark'', ``lark'', ``bark'', and so on, viz. [\textipa{kwA:rk}] rather than the more common [\textipa{kwO:rk}]!}~\cite{gell-mann_schematic_1964}. These particles, along with the electron-like family of leptons (including neutrinos), the gauge bosons, and the Higgs boson, discovered in 2012~\cite{aad_observation_2012}, comprise the Standard Model, a mathematical model which describes all the known forces and matter of the universe, with the notable exceptions (at time of writing) of gravity, dark matter, dark energy, and neutrino masses.

This thesis begins at a time when physicists are working hard to find gaps in this model, mostly by probing higher and higher ranges of energy. The experimental work being done at GlueX, however, resides in a lower energy regime, which we usually describe as ``medium energy physics''. As I will elucidate later in this manuscript, the strong force is non-perturbative in this regime, making direct calculations through the Standard Model all but impossible. However, since the advent of Lattice Quantum Chromodynamics (LQCD) in 1974~\cite{wilson_confinement_1974}, physicists have been able to make approximate predictions via computer simulations of the theory.

\section{Thesis Overview}\label{sec:thesis_overview}

Herein, I will focus on a particular portion of the Standard Model that dictates the strong interaction, viz. interactions between quarks and gluons, the mediating gauge boson of the strong force. Beginning with a discussion of the theory and history of $K_S$ (K-short) pair production in prior experiments, I will give a brief overview of the GlueX experiment. I will then outline some of the theoretical underpinnings and implications of glueballs to persuade the reader on the importance of this production channel in the larger scheme of GlueX.

Next, I will describe my own analysis, beginning with the the impetus of this study, a search for $\Sigma^+$ baryons using a different recombination of the final state in this channel. This will lead to a first-order peek at the many resonances which decay to $K_S$ pairs, and I will delineate the layers of data selection which I have carried out to produce a clean sample of events.

I will then discuss the process of partial-wave analysis (PWA), modeling resonances, and selecting a waveset for my data. I will conclude with the results from fits of these models to the data, the implications of such fits, and the next steps which I or another future particle physicist might take in order to illuminate another corner of the light mesonic spectrum.

\section{Motivation}\label{sec:motivation}

While this will be discussed in detail later, I believe it is important to emphasize the motivation for such a study of photoproduction of $K_SK_S$. While the majority of GlueX research concerns the study of hybrid mesons (mesons with forbidden quantum numbers), such mesons cannot be found in this channel. Given a bound state of two spin-$\frac{1}{2}$ quarks with relative angular momentum $L$, total spin $S$ and total angular momentum $J$ (the eigenvalue of $\hat{J}^2 = \hat{L}^2 \oplus \hat{S}^2$), we can define the parity operator $\hat{P}$ by its effect on the wave function of the system,
\begin{equation}
  \hat{P}\ket{\vec{r}} = \eta\ket{\vec{r}}
\end{equation}
where $\eta$ can be determined by noting that states of angular momentum are generally proportional to a spherical harmonic in their angular distribution ($\ket{r,\theta,\varphi;LM} \sim Y_L^M(\theta,\varphi)$) and
\begin{equation}
  \hat{P}Y_L^M(\theta,\varphi) = Y_L^M(\pi-\theta,\pi+\varphi) = (-1)^LY_L^M(\theta,\varphi)
\end{equation}
so $\eta = (-1)^L$. The Dirac equation can be used to show that the intrinsic parity of quarks and antiquarks, when multiplied, yields a factor of $-1$, so
\begin{equation}
  \hat{P}\ket{q\bar{q};JLMS} = -(-1)^L
\end{equation}

Similarly, the operator $\hat{C}$ representing C-parity will also introduce a factor of $(-1)^L$ because exchanging charges of a (neutral\footnote{For $\hat{C}$ to be Hermitian, and thus observable, acting it twice on a state should return the original state, so only eigenvalues of $\pm 1$ are allowed. Therefore, only states which are overall charge neutral are eigenstates of $\hat{C}$.}) quark-antiquark system is akin to reversing their positions under parity. If $\ket{S}$ is antisymmetric under C-parity, we should get an additional factor of $-1$, which is the case for the $S=0$ singlet. With the aformentioned $-1$ due to the intrinsic parity of the quarks and antiquarks, we find
\begin{equation}
  \hat{C}\ket{q\bar{q};JLMS} = (-1)^{L+S}
\end{equation}

Labeling states with the common $J^{PC}$ notation, it can then be shown that states like $0^{--}$, $0^{+-}$, $1^{-+}$, and $2^{+-}$ (among others) are not allowed states for $q\bar{q}$ mesons. As mentioned, the investigation of such states is the primary focus of the GlueX experiment. However, since the particle we are concerned with decays to two identical particles ($K_S$) which have a symmetric spatial wave function, and because this particle is a meson which follows Bose-Einstein statistics, the angular part of the total wave function must also be symmetric, i.e. $J = \text{even integers}$. Furthermore, because parity is conserved in strong decays, and the state of two identical particles is symmetric under parity, the decaying meson must also have $P=+$. Finally, the strong interaction also conserves C-parity, and both kaons are neutral, so we can determine the $J^{PC}$ quantum numbers of the resonance to be $\text{even}^{++}$. There should be no overlap here with the aformentioned hybrid mesons, but that does not mean the channel is not of interest to GlueX and the larger scientific community. Particularly, the lowest lying glueball states are predicted to not only share these quantum numbers, but exist in the middle of the mass range produced by GlueX energies\cite{morningstar_glueball_1999}. To add to this, the spin-$0$ isospin-$0$ light flavorless mesons, denoted as $f_0$-mesons, are supernumerary, either due to mixing with a supposed light scalar glueball or by the presence of a light tetraquark (or both)\cite{particle_data_group_review_2020}.

However, it would be an understatement to say that the $K_SK_S$ channel at GlueX is not the ideal place to be looking for either glueballs or tetraquarks. This is because, while we have excellent handles for reconstructing this channel, we have no ability to separate particles of different isospin with these data alone. This means that these $f$ states will be indistinguishable from their isospin-$1$ partners, the $a$-mesons. At first glance, it might seem like a model of the masses of these particles would make it easy to separate them, even if they remained indistinguishable between resonant peaks, but with broad states like the $f_0(1370)$ and states which sit right on top of each other (like the $f_0(980)$ and $a_0(980)$, which also tend to interfere with each other), there is likely no unique mass model which can distinguish all of the possible states without relying on data from other channels.

The silver lining is that, due to the GlueX detector's state-of-the-art angular acceptance\cite{adhikari_gluex_2021}, we do stand a chance at separating spin-$0$ states from spin-$2$ states, and GlueX's polarized beam allows us to further understand the mechanisms at play by giving us some indication of the parity of the $t$-channel exchanged particle in the production interaction. We can also use this channel as a proving ground for more complex amplitude analysis involving a mass model, which could be extended to a coupled-channel analysis in the future.


\section{Past Analyses}
This is certainly not the first attempt to study $K_S^0$ pair production. The earliest published experiment with a similar final state dates back to 1961, where researchers at CERN measured 54 events which ended in a final state which included two neutral kaons. Since only $K_S^0$ eigenstates decayed inside the bubble chamber, several of these early experiments infered a final state of $K_S^0K_S^0$. For the majority of the 1960s and 1970s, research into this final state was dominated by pion beam production off a proton target, aside from a kaon beam experiment in 1977 (see \Cref{tab:past-analyses}). In the 1980s, several collaborations at DESY began studying electron-positron collisions, which imply an internal virtual photon fusion as the production mechanism. These experiments are an important counterpoint to those involving hadrons, since the glueball does not couple to photons, so any intermediate glueball production in these reactions should be heavily suppressed~\cite{acciarri_k0sk0s_2001}, although radial excitations should couple~\cite{mathieu_physics_2009}. While the statistical power of these experiments was very small at first (relative to pion beam experiments), the L3 collaboration at LEP and Belle at KEKB provided larger data samples in the first two decades of the 2000s. Until this study, ITEP and BNL held the statistical lead in non-photon-fusion experiments, and we now present a dataset which is at least 2.5 times as large as either.

There has only been one prior experiment, published by the CLAS collaboration in 2018~\cite{the_clas_collaboration_double_2018}, which used photoproduction as a means of accessing this channel. While the ``golden channel'' for glueball production remains radiative $J/\psi$ decays to $K_S^0K_S^0\gamma$, since non-glueball intermediate processes converting charm quarks into strange quarks are suppressed, photoproduction could be effective at creating exotic hybrid states as well as glueballs due via the photon's hadronic component or Pomeron exchange~\cite{mathieu_physics_2009}. Additionally, the CLAS experiment did not utilize the polarized photon beam capability at JLab, but the current analysis at GlueX does, and this access to linear photon polarization can inform us of the parity exchanged in production of such exotic states.

\begin{table}
  \begin{center}
    \begin{tabular}{cccc}\toprule
      Channel & Collaboration & \# Events & Year\\\midrule
      $\pi^- p \to K_S^0 K_S^0 n$ & CERN/PS & 54 & 1961~\cite{cremieu-alcan_proceedings_1962}\\
      $\pi^- p \to K_S^0 K_S^0 n$ & BNL & 19 & 1962~\cite{erwin_experimental_1962}\\
      $\pi^- p \to K_S^0 K_S^0 n$ & LRL & 66 & 1962~\cite{alexander_final-state_1962}\\
      $\pi^- p \to K_S^0 K_S^0 + \text{neutrals}$ & BNL & 374 & 1966~\cite{crennell_observation_1966}\\
      $\pi^- p \to K_S^0 K_S^0 n$ & LRL & 426 & 1966~\cite{hess_low-mass_1966}\\
      $\pi^- p \to K_S^0 K_S^0 n$ & LRL & 418 & 1967~\cite{dahl_strange-particle_1967}\\
      $\pi^- p \to K_S^0 K_S^0 n$ & CERN/PS & 2559 & 1967~\cite{beusch_observation_1967}\\
      $\pi^- p \to K_S^0 K_S^0 n$ & ANL/ZGS & 1969 & 1968~\cite{hoang_k_1_0k_1_0_1968} \& 1969~\cite{hoang_investigation_1969}\\
      $\pi^- p \to K_S^0 K_S^0 n$ & CERN/PS & 4820 & 1975~\cite{beusch_new_1975}\\
      $\pi^- p \to K_S^0 K_S^0 n$ & CERN/PS & 6380 & 1976~\cite{wetzel_study_1976}\\
      $\pi^- p \to K_S^0 K_S^0 n$ & ANL/ZGS & 5096 & 1976~\cite{cason_observation_1976} \& 1979~\cite{polychronakos_study_1979}\\
      $K^- p \to K_S^0 K_S^0 + \text{neutrals}$ & CERN/PS & 410 & 1977~\cite{barreiro_production_1977}\\
      $\pi^- p \to K_S^0 K_S^0 n$ & BNL/MPS & 1278 & 1980~\cite{gottesman_peripheral_1980}\\
      $\pi^- p \to K_S^0 K_S^0 n$ & BNL/MPS & 29381 & 1982~\cite{etkin_amplitude_1982}\\
      $e^+e^- \to e^+e^- K_S^0 K_S^0$ & DESY/TASSO & 100 & 1983~\cite{althoff_production_1983}\\
      ??? & ??? & ??? & 1986 (Bolonkin)\\
      ??? & ??? & ??? & 1986 (Baloshin)\\
      $\pi^- p \to K_S^0 K_S^0 n$ & BNL/MPSII & 40494 & 1986~\cite{longacre_measurement_1986}\\
      $e^+e^- \to e^+e^- K_S^0 K_S^0$ & DESY/PLUTO & 21 & 1987~\cite{berger_tensor_1988}\\
      $K^- p \to K_S^0 K_S^0 \Lambda$ & SLAC/LASS & 441 & 1988~\cite{aston_study_1988}\\
      $e^+e^- \to e^+e^- K_S^0 K_S^0$ & DESY/CELLO & 26 & 1988~\cite{behrend_theks0ks0final_1989}\\
      $e^+e^- \to e^+e^- K_S^0 K_S^0$ & LEP/L3 & 62 & 1995~\cite{acciarri_study_1995}\\
      $pp \to pp K_S^0 K_S^0$ & Fermilab/E690 & 11182 & 1998~\cite{reyes_partial_1998}\\
      $\pi^- p \to K_S^0 K_S^0 + \text{neutrals}$ & ITEP & 1000${}^\dagger$ & 1999~\cite{barkov_discovery_1999}\\
      $e^+e^- \to e^+e^- K_S^0 K_S^0$ & LEP/L3 & 802 & 2001~\cite{acciarri_k0sk0s_2001}\\
      $\pi^- C \to K_S^0 K_S^0 + X$ & ITEP & 553 & 2003~\cite{tikhomirov_resonances_2003}\\
      $e^+e^- \to e^+e^- K_S^0 K_S^0$ & LEP/L3 & 870 & 2006~\cite{schegelsky_k0_2006}\\
      $\pi^- p \to K_S^0 K_S^0 n$ & ITEP & 40553 & 2006~\cite{vladimirsky_analysis_2006}\\
      $e^+e^- \to e^+e^- K_S^0 K_S^0$ & KEKB/Belle & 900000${}^\ddagger$ & 2013~\cite{uehara_high-statistics_2013}\\
      $\gamma p \to K_S^0 K_S^0 p$ & JLAB/CLAS & 13500${}^\dagger$ & 2018~\cite{the_clas_collaboration_double_2018}\\
      $\gamma p \to K_S^0 K_S^0 p$ & JLAB/GlueX & 102331${}^\ast$ & 2025 (this analysis)\\\bottomrule
    \end{tabular}
    \caption{Summary of all (known) past analyses involving production of $K_S^0$ pairs. Note that this is possibly not exhaustive and does not include any studies which focus on decays of an intermediate state, i.e. $J/\psi \to \gamma K_S^0K_S^0$.\newline$\dagger$ - Estimated from plots.\newline$\ddagger$ - Reported as three orders of magnitude larger than LEP's result from 2006, but exact count was difficult to estimate.\newline$\ast$ - Weighted number of events with $\chi^2_\nu < 3.0$.}\label{tab:past-analyses}
  \end{center}
\end{table}
