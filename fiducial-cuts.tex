To get a clearer picture of the data, we will first make a loose cut selecting only events with $\chi^2_\nu < 10$. We can then generate phase-space Monte Carlo for the $4\pi$ background reaction and search for potential ways to separate it from the signal by comparing the distributions of common kinematic variables.

\begin{figure}
  \begin{center}
    \includesvg[width=0.8\textwidth]{ext/analysis/plots/chisqdof_combined_None_None_None.svg}
  \end{center}
  \caption{The normalized distributions of $\chi^2_\nu$ for the true data and phase-space signal and $4\pi$ background Monte Carlo.}\label{fig:data-combined-chisqdof}
\end{figure}

In \Cref{fig:data-combined-chisqdof}, we can see the various distributions of the data and each simulated channel. Unfortunately, the data distribution resembles the $4\pi$ background distribution much better than the signal Monte Carlo, but now we also know that the signal decays rapidly at higher values of $\chi^2_\nu$, so a tighter selection might improve the signal-to-background ratio, even if we cannot immediately see a peak in the data near unity. It is difficult to choose a value at which to cut in this variable, but the choice is largely arbitrary. We will be using a statistical weighting method in \Cref{sec:splot} to subtract the non-strange background represented by the $4\pi$ Monte Carlo, but we must make some choice on $\chi^2_\nu$ before that step in the analysis to reduce other backgrounds mentioned in \Cref{sec:data-selection}. For simplicity, we will conduct the entire analysis using a selection of $\chi^2_\nu < 3.0$, and we will later select several other values on either side of this cut to study its systematic effect in \Cref{sec:systematic-studies}.

As mentioned in \Cref{sec:data-selection}, the asymmetry in missing energy in \Cref{fig:me-combined} likely comes from background channels with photons which were missed in reconstruction. The selection on $\chi^2_\nu$ greatly reduces the contributions from these backgrounds, as can be seen in \Cref{fig:me-combined-chisqdof-3.0}

\begin{figure}
  \begin{center}
    \includesvg[width=0.8\textwidth]{ext/analysis/plots/me_combined_pz_masscut_chisqdof_3.0_mesons_None_None.svg}
  \end{center}
  \caption{Normalized distributions of the missing energy ($E_i - E_f$) for data, phase-space signal Monte Carlo, and $4\pi$ background Monte Carlo after a selection of $\chi^2_\nu < 3.0$ is applied. {\color{red}TODO: get plot without other fiducial cuts}}\label{fig:me-combined-chisqdof-3.0}
\end{figure}


\begin{figure}
  \begin{center}
    \includesvg[width=0.8\textwidth]{ext/analysis/plots/mm2_combined_pz_masscut_chisqdof_3.0_mesons_None_None.svg}
  \end{center}
  \caption{The normalized distributions of missing mass squared for the true data and phase-space signal and $4\pi$ background Monte Carlo after a selection of $\chi^2_\nu < 3.0$ is applied. {\color{red}TODO: get plot without other fiducial cuts, make this a log-plot or change bounds}}\label{fig:mm2-combined-chisqdof-3.0}
\end{figure}

Another common selection which can be made is on the squared missing mass of an event, which is just the square of the difference between the initial- and final-state four-momenta. This variable, visualized in \Cref{fig:mm2-combined-chisqdof-3.0}, predictably peaks at zero for the data, signal Monte Carlo, and background Monte Carlo, and the overall distributions for the signal and backgound simulated events are nearly identical. This indicates that a selection on this variable would not be very useful at distinguishing the signal from the background.

There are two more minor selections which we will perform on the data. The first is a selection on the invariant mass of $K_S^0K_S^0$ at $\text{IM} < \SI{2.0}{\giga\electronvolt}$. This is done because models we use later to describe the data do not include any higher-mass resonances, and data past this point causes issues with the mass-dependent fits in \Cref{sec:mass-dependent-fits}. The other is a selection on the $z$-vertex of the target proton. We know the exact location of the target with respect to the detector, and we only want to deal with events which we know originated inside this target. The distribution of the $z$-vertex values can be seen in \Cref{fig:protonz-combined}. We will select events with $\SI{50}{\centi\meter} < z < \SI{80}{\centi\meter}$, as this is the known length and position of the target relative to the detector elements. While there is an indication of events in the signal Monte Carlo in the regions which are removed, we know that while these events originated from the target, their $z$-vertex has been misidentified, so we should remove them anyway.


\begin{figure}
  \begin{center}
    \includesvg[width=0.8\textwidth]{ext/analysis/plots/protonz_combined_None_None_None.svg}
  \end{center}
  \caption{The normalized distributions of the target proton $z$-vertex for the true data and phase-space signal and $4\pi$ background Monte Carlo (no cuts applied).}\label{fig:protonz-combined}
\end{figure}

An impetus for studying this channel was originally a search for excited hyperons\textemdash baryons with strange quark content. In particular, $\Sigma^+$ resonances should form in the invariant mass distribution of $K_S^0 p$, where the other kaon is required to conserve strangeness in the strong production of this baryon. There is a symmetry between the identical kaons which must first be accounted for. We could pair either kaon with the proton to form the $\Sigma^+$, but there is a more likely pairing which can be found by first recognizing that the heavier hyperon will generally move more backwards in the center-of-momentum frame than the lighter bachelor $K_S^0$. Therefore, the kaon which is moving in a further backward direction is more likely to correspond with the hyperon decay candidate. We can sort the kaons by their $\cos\theta$ in the center-of-momentum frame and call the more backward-going kaon $K_{S,B}^0$. The invariant mass spectrum of $K_{S,B}^0 p$ can be seen in \Cref{fig:baryon-mass-data-pz-masscut-chisqdof-3.0}. The enhancement around $\SI{1.7}{\giga\electronvolt}$ likely corresponds to some combination of the $\Sigma^+(1660)$, $\Sigma^+(1670)$, $\Sigma^+(1750)$, and $\Sigma^+(1775)$ resonances. We do not see much indication of states above $\SI{2}{\giga\electronvolt}$.

\begin{figure}
  \begin{center}
    \includesvg[width=0.8\textwidth]{ext/analysis/plots/baryon_mass_data_pz_masscut_None_None_None.svg}
  \end{center}
  \caption{The mass distribution of the backward-going $K_S^0$ combined with the proton. {\color{red}TODO: get the proper cuts}}\label{fig:baryon-mass-data-pz-masscut-chisqdof-3.0}
\end{figure}


This angle also gives us a handle on separating baryonic and mesonic topologies. If we plot the $\cos\theta$ of $K_{S,B}^0$ against the invariant mass of $K_{S,B}^0 p$, as in \Cref{fig:ksb-costheta-v-baryon-mass-data-pz-masscut-chisqdof-3.0}, we find a strong and somewhat expected dependence. The majority of the baryonic contribution occurs at angles with $\cos\theta < 0$. Likewise, the mesonic contribution is mostly confined to angles of $\cos\theta > 0$, as seen in figures \Cref{fig:ksb-costheta-v-meson-mass-data-pz-masscut-chisqdof-3.0,fig:meson-mass-data-pz-masscut-chisqdof-3.0-mesons}. While it may not be obvious that there even are meson resonances here, we will see their contributions much more clearly after the statistical weighting procedures defined in \Cref{sec:splot} have been applied, after which we will revisit these plots. We will also conduct our analysis without this selection in \Cref{sec:systematic-studies} for completeness. We can also isolate the baryon contributions by reversing this selection, and the result of this is seen in \Cref{fig:baryon-mass-data-pz-masscut-chisqdof-3.0-baryons}.

\begin{figure}
  \begin{center}
    \includesvg[width=0.8\textwidth]{ext/analysis/plots/ksb_costheta_v_baryon_mass_data_masscut_None_None_None.svg}
  \end{center}
  \caption{The center-of-momentum azimuthal angle of $K_{S,B}^0$ plotted against the invariant mass of $K_{S,B}^0 p$. The baryonic contribution occurs mostly at far backward angles, while mesons appear at the far forward angles. {\color{red}TODO: get the proper cuts}}\label{fig:ksb-costheta-v-baryon-mass-data-pz-masscut-chisqdof-3.0}
\end{figure}

\begin{figure}
  \begin{center}
    \includesvg[width=0.8\textwidth]{ext/analysis/plots/ksb_costheta_v_meson_mass_data_masscut_None_None_None.svg}
  \end{center}
  \caption{The center-of-momentum azimuthal angle of $K_{S,B}^0$ plotted against the invariant mass of $K_S^0K_S^0$. The baryonic contribution occurs mostly at far backward angles, while mesons appear at the far forward angles. {\color{red}TODO: get the proper cuts}}\label{fig:ksb-costheta-v-meson-mass-data-pz-masscut-chisqdof-3.0}
\end{figure}

\begin{figure}
  \begin{center}
    \includesvg[width=0.8\textwidth]{ext/analysis/plots/meson_mass_data_pz_masscut_chisqdof_3.0_mesons_None_None.svg}
  \end{center}
  \caption{The invariant mass of $K_S^0K_S^0$ across all datasets after all fiducial selections are applied.}\label{fig:meson-mass-data-pz-masscut-chisqdof-3.0-mesons}
\end{figure}

% \begin{figure}
%   \begin{center}
%     \includesvg[width=0.8\textwidth]{ext/analysis/plots/baryon_mass_data_pz_masscut_chisqdof_3.0_baryons_None_None.svg}
%   \end{center}
%   \caption{The invariant mass of $K_S^0K_S^0$ across all datasets after all fiducial selections are applied, but with the baryon rejection cut reversed to select baryons.}\label{fig:baryon-mass-data-pz-masscut-chisqdof-3.0-baryons}
% \end{figure}

In total, we have the fiducial selections on the data illustrated in \Cref{tab:fiducial-cuts}.

\begin{table}
  \begin{center}
    \begin{tabular}{cc}\toprule
      Variable & Selected Values \\\midrule
      $\chi^2_\nu$ & $\chi^2_\nu < 3.0$ \\
      Mass of $K_S^0K_S^0$ & $ m < \SI{2.0}{\giga\electronvolt} $ \\
      Target proton-$z$ & $\SI{50}{\centi\meter} < z < \SI{80}{\centi\meter}$ \\
      $\cos\theta_{\text{CM}}$ of $K_{S,B}^0$ & $ \cos\theta_{\text{CM}} > 0.0 $ \\\bottomrule
    \end{tabular}
    \caption{Fiducial cuts performed after event reconstruction.}\label{tab:fiducial-cuts}
  \end{center}
\end{table}

{\color{red}TODO: dE/dx plots in CDC, FDC for $p$/$\pi$ separation (appendix?)}
