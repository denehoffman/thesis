\section{The GlueX Experiment}\label{sec:the-gluex-experiment}
GlueX, short for the Gluonic Excitation experiment located at the Thomas Jefferson National Accelerator Facility (JLab), first began collecting publication-ready data in 2016 with the goal of establishing the spectrum of light exotic states such as hybrid mesons and glueballs.

GlueX recieves an unpolarized electron beam from the Continuous Electron Beam Accelerator Facility (CEBAF), which is then converted to linearly polarized photons via coherent bremsstrahlung from a diamond radiator. The scattering electrons are detected in an array of high-resolution scintillators called the Tagger Microscope (TAGM) which covers beam energies between $8$ and $\SI{9}{\giga\eV}$, a region of energy refered to as the ``coherent peak''. This range of beam energies has been specifically tuned (by modification of the radiator properties) to provide the highest amount of polarization{\color{red}[FIGURE]}. The rest of the energy range, regions from about $3$--$\SI{8}{\giga\eV}$ and $9$--$\SI{12}{\giga\eV}$, are covered by the lower-resolution Tagger Hodoscope (TAGM). These taggers are used to determine the photon energy, since it is equal to the difference between the incident and outgoing electron energies~\cite{adhikari_gluex_2021}.

The photon beam next passes through a thin beryllium foil which induces $e^+e^-$ pair production. The azimuthal angle of the produced electrons is measured by the Triplet Polarimeter (TPOL) to measure the photon polarization. The Pair Spectrometer (PS) then counts these electrons in coincidence with the TAGM and TAGH detectors. A known fraction of the beam is converted to $e^+e^-$ pairs in this way, allowing for an accurate measurement of the photon flux and the energy dependence of the polarization fraction.

Next, the photon interacts with a liquid hydrogen cryotarget, which maintains a temperature of $\SI{20.1}{\K}$, allowing the contents to act as a stationary proton target. A set of thin scintillators called the Start Counter (ST) surrounds the target, which captures the initial signals (and azimuthal angles) from reaction products and associates them with the electron radio frequency (RF) beam bunch from which the reaction originated. Any products of the reaction next pass through either the Central Drift Chamber (CDC) or Forward Drift Chamber (FDC) depending on their trajectory (particles moving along the direction of the beamline pass through the FDC). These chambers are filled with wires which sense the fields generated by charged particles as they pass nearby, allowing for the reconstruction of charged trajectories. The entire detector is situated inside a solenoid with a magnetic field around $\SI{2}{\tesla}$, and this field bends the trajectories of charged particles, allowing for proper identification of the sign of the charge as well as the particle's momentum. These chambers also allow for reconstruction of decay vertices by tracing back charged trajectories to their nearest approach.

The Barrel Calorimeter (BCAL) surrounding the CDC and the Forward Calorimeter (FCAL) in front of the FDC measure the energy of photon showers from interactions with charged and neutral particles. This includes photons originating from the decays of light neutral mesons like the $\pi^0$ and $\eta$. Because such neutral particles pass through the CDC and FDC without detection, these calorimeters are necessary for their reconstruction. However, since only three vertices exist in such reconstructions (the initial vertex in the target and the final vertices in the calorimeter), GlueX is unable to reconstruct the decay vertices of these neutral particles. Since we are studying the a reaction with a fully charged final state, this might not seem important. However, in {\color{red}Section ?}, we will discuss the study of neutral decays of $K_S^0\to 2\pi^0 \to 4\gamma$, where the reconstruction will be severely limited due to the inability to reconstruct the decay vertex of the $K_S^0$.

There is a final scinatillator called the Time-of-Flight (TOF) detector situated immediately between the FDC and the FCAL. In combination with the ST, this aids in charged-particle identification via a measurement of the flight time between the two detectors. Additionally, in 2019, an additional detector, the DIRC\footnote{An acronym for Detection of Internally Reflected Cherenkov light} detector was installed between the TOF and FDC to further aid in identification of charged pions and kaons. This detector measures the angle of the cone of Cherenkov radiation emitted by relativistic charged particles, which can be used to ascertain their velocity via the relation $\cos\theta_c \sim 1/v$. When compared to measurements of the particle's momentum, this detector can be used to distinguish pions, kaons, and protons. Further information about the GlueX detector and the DIRC can be found in references \cite{adhikari_gluex_2021} and \cite{ali_installation_2020} respectively.

The data analyzed in this thesis were collected in four separate ``run periods'' across two experiment ``phases'', denoted Phase-I and Phase-II. Phase-I consists of three run periods, notated Spring 2017, Spring 2018, and Fall 2018 by the season and year when data collection began, and Phase-II contains one run period, Spring 2020\footnote{This run technically began at the end of 2019} and is the only dataset which was collected after the DIRC installation. The total luminosity, coherent peak range, and luminosity in the coherent peak for each run period is listed in \Cref{tab:run-info}.


\begin{table}
  \begin{center}
    \begin{tabular}{ccccc}\toprule
      Experiment & Run Period & Luminosity ($E_\gamma > \SI{6.0}{\giga\eV}$) & Coherent Peak Range & Luminosity in Coherent Peak \\\midrule
      Phase-I & Spring 2017 & $\SI{74.7}{\pico\barn^{-1}}$ & $8.2$--$\SI{8.8}{\giga\eV}$ & $\SI{21.8}{\pico\barn^{-1}}$ \\
              & Spring 2018 & $\SI{223.8}{\pico\barn^{-1}}$ & $8.2$--$\SI{8.8}{\giga\eV}$ & $\SI{63.0}{\pico\barn^{-1}}$ \\
              & Fall 2018 & $\SI{141.1}{\pico\barn^{-1}}$ & $8.2$--$\SI{8.8}{\giga\eV}$ & $\SI{40.1}{\pico\barn^{-1}}$ \\\midrule
      Phase-II & Spring 2020 & $\SI{386.2}{\pico\barn^{-1}}$ & $8.0$--$\SI{8.6}{\giga\eV}$ & $\SI{132.4}{\pico\barn^{-1}}$ \\\midrule
      Total & & $\SI{825.8}{\pico\barn^{-1}}$ & & $\SI{257.3}{\pico\barn^{-1}}$ \\\bottomrule
    \end{tabular}
    \caption{Summary of total luminosity and total luminosity in the coherent peak for each run period.}\label{tab:run-info}
  \end{center}
\end{table}

The entire GlueX detector has been simulated with Geant4. Due to small changes and updates to the detector and its simulation between run periods, we will treat these datasets separately during the analysis, only presenting combined results in summary. Therefore, when generating Monte Carlo simulated data to model detector acceptance, each run period must be simulated separately.

\subsection{Particle Identification and the GlueX Kinematic Fit}\label{sub:particle-identification-and-the-gluex-kinematic-fit}
Data collected from the GlueX detector mostly contains reaction topologies (the set of initial\-/, intermediate\-/, and final\-/state particles) which are not the channel of interest in this thesis, as the GlueX experiment uses a broad trigger to collect data with a wide range of final states. Additionally, due to the unavoidable finite resolution of each detector, the measured quantities such as the momentum of each particle might not exactly align with physical expectations (such as four-momentum conservation). For these reasons, we need to filter the detected data to just events which match our desired topology and kinematically fit the observables such as particle masses, decay vertices, and four-momenta with constraints that enforce conservation laws and other properties of interest.

This process begins with particle reconstruction, where the raw detector data is transformed into charged particle tracks (from the drift chambers), electromagnetic showers (from the calorimeters), and timing data from the ST, TOF, and accelerator RF signal. Next, the charged tracks are matched up with their respective showers from the calorimeters, and the remaining showers are labeled as ``neutral showers'' originating from photons (or possibly neutral particles like neutrons). At this stage, no particle identification is assigned to the charged and neutral tracks. Track reconstruction is performed before data is ``skimmed'' (separated into final-state topologies) so that all GlueX datasets are processed in the same way. My own analysis and data selection happens after track reconstruction, particle identification cuts, and the kinematic fit, after which there are still many events which likely belong to backgrounds which make it through these steps.

\subsubsection{Particle Identification Cuts}

The next stage involves filtering through reconstructed tracks to find candidate events which match our topology. Timing cuts are applied to whichever detector gives the best timing information (the order being BCAL, TOF, FCAL, ST). By timing, we mean the difference between the time measured in the detector and the time of the RF beam bunch measured in the TAGH/TAGM. Each set of charged tracks is identified with each charged particle in the topology (events with too many or too few charged tracks are excluded). Each hypothetical identification is then subjected to cuts on the energy lost in the drift chambers or deposited in the calorimeters. Since there are no ``missing'' final-state particles (such as a neutron) in our reaction, an additional cut is made on the missing energy (the difference between the initial energy from the beam photon and the summed energy of the detected final-state particles). Finally, cuts are made on the invariant mass of some particles, as well as the {\color{red}TODO} missing mass (squared) of the entire reaction (the squared difference in invariant mass between the initial and final state). A summary of these particle identification (PID) cuts can be seen in \Cref{tab:pid-cuts}.

\begin{table}
  \begin{center}
    \begin{tabular}{cccc}\toprule
      Particle & Selected Values & Unit & Detector \\\midrule
      $\gamma$ & $-1.5 \leq \Delta t_{\text{RF}} \leq 1.5$ & $\si{\nano\s}$ & BCAL\\
             & $-2.5 \leq \Delta t_{\text{RF}} \leq 2.5$ & $\si{\nano\s}$ & FCAL\\
             & $-0.1 \leq \text{MM}^2 \leq 0.1$ & $\si{(\giga\eV/c^2)^2}$ & N/A\\\midrule
      $\pi^{\pm}$ & $-1.0 \leq \Delta t_{\text{RF}} \leq 1.0$ & $\si{\nano\s}$ & BCAL\\
                & $-0.5 \leq \Delta t_{\text{RF}} \leq 0.5$ & $\si{\nano\s}$ & TOF\\
                & $-2.0 \leq \Delta t_{\text{RF}} \leq 2.0$ & $\si{\nano\s}$ & FCAL\\
                & $-2.5 \leq \Delta t_{\text{RF}} \leq 2.5$ & $\si{\nano\s}$ & ST\\
                & $\dv{E}{x} < \exp[-7.0\abs{\vec{p}} + 3.0] + 6.2$ & $\si{\kilo\eV/\centi\m}$ ($\dv{E}{x}$), $\si{\giga\eV/c}$ ($\abs{\vec{p}}$) & CDC \\
                & $-1.0 \leq \text{MM}^2 \leq 1.0$ & $\si{(\giga\eV/c^2)^2}$ & N/A\\\midrule
      $p$ & $-1.0 \leq \Delta t_{\text{RF}} \leq 1.0$ & $\si{\nano\s}$ & BCAL\\
                & $-0.6 \leq \Delta t_{\text{RF}} \leq 0.6$ & $\si{\nano\s}$ & TOF\\
                & $-2.0 \leq \Delta t_{\text{RF}} \leq 2.0$ & $\si{\nano\s}$ & FCAL\\
                & $-2.5 \leq \Delta t_{\text{RF}} \leq 2.5$ & $\si{\nano\s}$ & ST\\
                & $\dv{E}{x} > \exp[-4.0\abs{\vec{p}} + 2.25] + 1.0$ & $\si{\kilo\eV/\centi\m}$ ($\dv{E}{x}$), $\si{\giga\eV/c}$ ($\abs{\vec{p}}$) & CDC \\
                & $-0.5 \leq \text{MM}^2 \leq 4.41$ & $\si{(\giga\eV/c^2)^2}$ & N/A\\\midrule
      $\pi^0$ & $0.08 \leq \text{IM} \leq 0.19$ & $\si{\giga\eV/c^2}$ & N/A\\
              & $-1.0 \leq \text{MM}^2 \leq 1.0$ & $\si{(\giga\eV/c^2)^2}$ & N/A\\\midrule
      $K_S^0$ & $0.3 \leq \text{IM} \leq 0.7$ & $\si{\giga\eV/c^2}$ & N/A\\
              & $-1.0 \leq \text{MM}^2 \leq 2.0$ & $\si{(\giga\eV/c^2)^2}$ & N/A\\\midrule
      N/A & $-3.0 \leq \text{ME} \leq 3.0$ & $\si{\giga\eV}$ & N/A\\\bottomrule
    \end{tabular}
    \caption{PID cuts used in event reconstruction. $\text{MM}^2$ corresponds to the missing mass squared, $\text{IM}$ corresponds to the invariant mass, and $ME$ corresponds to the total missing energy.}\label{tab:pid-cuts}
  \end{center}
\end{table}

\subsubsection{Kinematic Fit}\label{subsub:kinematic-fit}

The final stage of reconstruction invokes a kinematic fit (KinFit) over data which has passed all separate PID cuts. The general structure of the kinematic fit is underpinned by the following derivation from {\color{red}[?]\footnote{How to cite GlueX docdb 2112?}}. This kinematic fit minimizes the following objective function:

\begin{equation}
  \chi^2(\vec{\eta},\vec{\xi}) = (\vec{y} - \vec{\eta})^\intercal \mathbf{V}_{\vec{y}}^{-1}(\vec{y} - \vec{\eta}) + 2 \vec{\lambda}^\intercal\vec{f}(\vec{\eta},\vec{\xi})
  \label{eq:kinfit-chi}
\end{equation}
where $\vec{y}$ are the experimentally measured values of observables $\vec{\eta}$, $\mathbf{V}_{\vec{y}}$ is the covariance matrix of the measured $\vec{y}$, $\vec{f}(\vec{\eta},\vec{\xi}) = 0$ are constraints applied to the system with additional free parameters $\vec{\xi}$ (which do not correspond to measured observables), and $\lambda$ are unknown Lagrange multipliers for said constraints. For example, $\vec{y}$ could include the measured four-momenta of all initial- and final-state particles, and $\mathbf{V}_{\vec{y}}$ would describe the uncertainty (from the detector elements) of each measurement. One possible constraint function $\vec{f}$ might minimize the difference between the initial and final four-momentum, as in \Cref{eq:four-momentum-constraint}. Since we wish to minimize $\chi^2$, we first take derivatives with respect to each set of free parameters,

\begin{align}
  \pdv{\chi^2}{\vec{\eta}} &= \mathbf{V}_{\vec{y}}^{-1}(\vec{\eta}-\vec{y}) + \left(\pdv{\vec{f}}{\vec{\eta}}\right)^\intercal\vec{\lambda} \label{eq:kinfit-dxdeta}\\
  \pdv{\chi^2}{\vec{\lambda}} &= \vec{f}(\vec{\eta},\vec{\xi}) \label{eq:kinfit-dxdlambda}\\
  \pdv{\chi^2}{\vec{\xi}} &= \left(\pdv{\vec{f}}{\vec{\xi}}\right)^\intercal\vec{\lambda} \label{eq:kinfit-dxdxi}
\end{align}

To find an extremum, we set these each to $\vec{0}$ and solve. At GlueX, the KinFit uses the Newton-Raphson method. First, we Taylor expand the constraint equations to first order,

\begin{equation}
  \vec{f}(\vec{\eta},\vec{\xi}) \approx \vec{f}(\vec{\eta}_0,\vec{\xi}_0) + \left(\pdv{\vec{f}}{\vec{\eta}}\right)\eval_{\vec{\eta}_0,\vec{\xi}_0}\left(\vec{\eta} - \vec{\eta}_0\right) + \left(\pdv{\vec{f}}{\vec{\xi}}\right)\eval_{\vec{\eta}_0,\vec{\xi}_0}\left(\vec{\xi} - \vec{\xi}_0\right)
\end{equation}

Assuming $\vec{\eta}_0$ and $\vec{\xi}_0$ are near the true minimum, we can set up an iterative method of approach,

\begin{equation}
  \vec{f}_i + \left(\pdv{\vec{f}}{\vec{\eta}}\right)_i\left(\vec{\eta}_{i+1} - \vec{\eta}_i\right) + \left(\pdv{\vec{f}}{\vec{\xi}}\right)_i\left(\vec{\xi}_{i+1} - \vec{\xi}_i\right) = \vec{0}
  \label{eq:kinfit-iter}
\end{equation}

where $\vec{f}_i \equiv \vec{f}(\vec{\eta}_i,\vec{\xi}_i)$, $\left(\pdv{\vec{f}}{\vec{\eta}}\right)_i\equiv \left(\pdv{\vec{f}}{\vec{\eta}}\right)\eval_{\vec{\eta}_i,\vec{\xi}_i}$, and $\left(\pdv{\vec{f}}{\vec{\xi}}\right)_i\equiv \left(\pdv{\vec{f}}{\vec{\xi}}\right)\eval_{\vec{\eta}_i,\vec{\xi}_i}$. To carry out an iteration, we need to determine the next step in each of the free directions $\vec{\eta}_{i+1}$ and $\vec{\xi}_{i+1}$. We start by writing the iterative forms of \Cref{eq:kinfit-dxdeta,eq:kinfit-dxdxi} as

\begin{align}
  \mathbf{V}_{\vec{y}}^{-1}(\vec{\eta}_{i+1} - \vec{y}) + \left(\pdv{\vec{f}}{\vec{\eta}}\right)^\intercal_i \vec{\lambda}_{i+1} &= 0 \label{eq:kinfit-dxdeta-iter} \\
  \left(\pdv{\vec{f}}{\vec{\xi}}\right)^\intercal_i \vec{\lambda}_{i+1} &= 0 \label{eq:kinfit-dxdxi-iter}
\end{align}

To obtain the next iterative estimate of $\vec{\eta}$ by rearranging \Cref{eq:kinfit-dxdeta-iter} as

\begin{equation}
  \vec{\eta}_{i+1} = \vec{y} - \mathbf{V}_{\vec{y}}\left(\pdv{\vec{f}}{\vec{\eta}}\right)^\intercal_i \vec{\lambda}_{i+1} \label{eq:kinfit-eta-update}
\end{equation}

Inserting this back into \Cref{eq:kinfit-iter}, we get

\begin{align}
  \vec{0} &= \underbrace{\vec{f}_i + \left(\pdv{\vec{f}}{\vec{\eta}}\right)_i\left(\vec{y}- \vec{\eta}_i\right)}_{\vec{r}_i} -\underbrace{\left(\left(\pdv{\vec{f}}{\vec{\eta}}\right)_i\mathbf{V}_{\vec{y}}\left(\pdv{\vec{f}}{\vec{\eta}}\right)^\intercal_i \right)}_{\mathbf{S}_i} \vec{\lambda}_{i+1} + \left(\pdv{\vec{f}}{\vec{\xi}}\right)_i\left(\vec{\xi}_{i+1} - \vec{\xi}_i\right) \notag\\
  \vec{0} &= \vec{r}_i - \mathbf{S}_i \vec{\lambda}_{i+1} + \left(\pdv{\vec{f}}{\vec{\xi}}\right)_i\left(\vec{\xi}_{i+1} - \vec{\xi}_i\right) \notag\\
  \vec{\lambda}_{i+1} &= \mathbf{S}_i^{-1}\left(\vec{r}_i + \left(\pdv{\vec{f}}{\vec{\xi}}\right)_i\left(\vec{\xi}_{i+1} - \vec{\xi}_i\right) \right) \label{eq:kinfit-lambda-update}
\end{align}

We can plug this definition of $\vec{\lambda}_{i+1}$ into \Cref{eq:kinfit-dxdxi-iter} to find the iterative update for $\vec{\xi}$:

\begin{align}
  \vec{0} &= \left(\pdv{\vec{f}}{\vec{\xi}}\right)^\intercal_i \mathbf{S}_i^{-1}\vec{r}_i + \underbrace{\left(\pdv{\vec{f}}{\vec{\xi}}\right)^\intercal_i \mathbf{S}_i^{-1}\left(\pdv{\vec{f}}{\vec{\xi}}\right)_i}_{\mathbf{U}_i}\left(\vec{\xi}_{i+1} - \vec{\xi}_i\right) \notag\\
  \vec{\xi}_{i+1} &= \vec{\xi}_i - \mathbf{U}_i^{-1}\left(\pdv{\vec{f}}{\vec{\xi}}\right)^\intercal_i \mathbf{S}_i^{-1}\vec{r}_i \label{eq:kinfit-xi-update}
\end{align}

To perform the kinematic fit, we use the available measured observables and their covariances (contained in $\vec{r}$, $\mathbf{S}$, and $\mathbf{U}$) to update our estimate of $\vec{\xi}$ via \Cref{eq:kinfit-xi-update}. Next, we use that result to update $\vec{\lambda}$ via \Cref{eq:kinfit-lambda-update}. Finally, we update $\vec{\eta}$ via \Cref{eq:kinfit-eta-update}. This process is repeated until the desired precision of these variables is met, and is guaranteed to converge as long as the initial values start near the minimum. The $\chi^2$ value obtained from evauating \Cref{eq:kinfit-chi} on the converged values tells us the quality of the fit. See \Cref{fig:chisqdof-initial} for the distribution of $\chi^2_\nu \equiv \chi^2 / \nu$, where $\nu$ is the number of degrees of freedom in the fit ($\nu=11$ for this analysis), for both the real data and Monte Carlo simulation of the signal.

\begin{figure}
  \begin{center}
    \begin{subfigure}[b]{.8\columnwidth}
      \includegraphics[width=1\linewidth]{figures/chisqdof_data_accpol.png}
      \caption{}
      \label{fig:chisqdof-data}
    \end{subfigure}
    \begin{subfigure}[b]{.8\columnwidth}
      \includegraphics[width=1\linewidth]{figures/chisqdof_accmc_accpol.png}
      \caption{}
      \label{fig:chisqdof-mc}
    \end{subfigure}
  \end{center}
  \caption{Distributions of $\chi^2_\nu$ (the $\chi^2$ per degree of freedom of the GlueX kinematic fit) for (a) data and (b) signal Monte Carlo after reconstruction cuts but before any fiducial selections from \Cref{sub:fiducial-cuts}. Note that the data extends past $\chi^2_\nu > 10$, but the plot was truncated for legibility. Both datasets have had accidental subtraction applied and combos flattened (see \Cref{subsub:combos,sec:data-selection}).}\label{fig:chisqdof-initial}
\end{figure}


The constraints $\vec{f}$ used in this channel include four-momentum conservation,

\begin{equation}
  \sum_{f\in\text{final}} p^\mu_{f} - \sum_{i\in\text{initial}} p^\mu_{i} = 0^\mu
  \label{eq:four-momentum-constraint}
\end{equation}

where the sum over four-momenta of of the initial-state particles must match that of the final-state particles, and the masses of each intermediate $K_S^0$,

\begin{equation}
  p_{K_S^0}^2 - m_{K_S^0}^2 = 0
\end{equation}

where $m_{K_S^0}$ is the known mass of the $K_S^0$ and $p_{K_S^0}$ is the fitted estimate of the four-momentum reconstructed from the $\pi^+\pi^-$ four-momentum measured by the detectors. Additionally we constrain the initial and final positional vertices of each production step in the topology. For this channel, this means we constrain the initial production vertex and the point of closest approach of the $K_S^0$s and recoil proton, as well as the decay vertex of each $K_S^0$ with the closest approach of its constituent pions. It is important to note that this vertex constraint cannot be done in the case where a $K_S^0$ decays to $2\pi^0$, since the neutral pions will decay to $2\gamma$ which can only be detected without tracking in the calorimeters. Because there is not a second point of detection for these photons, their trajectory and therefore the point at which the original $\pi^0$ decayed is unknown, so there is no way to further reconstruct the kaon decay vertex because the point of closest approach is unknown\footnote{The best we could do would be to assume the kaon decayed at the initial reaction vertex inside the target, which is not only unrealistic given that kaons decay via the weak interaction, but also causes problems later when we try to calculate the kaon lifetime to use in event weighting in \Cref{sec:splot}.}. Therefore, when we specify channels where one or both kaons decay as $K_S^0\to 2\pi^0$, we do not include these $K_S^0$s in the kinematic fit at all, and must infer them by adding the $\pi^0$ four-momenta (see \Cref{sub:neutral-kaon-decay-channels}). For the main $\gamma p \to K_S^0K_S^0p \to 2\pi^+2\pi^-p$ channel, we have four constraints from four-momentum conservation, two mass constraints (one for each kaon), 14 vertex constraints (two for each constrained particle\footnote{This is because we independently constrain the distance of a particle to the unknown vertex along two planes, the ``bend'' plane and the ``non-bend'' plane, named for being equal/perpendicular to the plane in which the trajectory of the particle is bending due to the magnetic field at the vertex point.}, viz. four pions, two kaons, and a proton), and nine unknowns from the vertex constraints (the position of the vertex introduces three unknowns for each vertex, and there are three constrained vertices, viz. two kaon decays and the initial production vertex). This gives a total of 20 constraints with nine unknowns, or 11 degrees of freedom in the fit.

\subsubsection{Combos}\label{subsub:combos}

While the initial interaction vertex in theory contains a single photon, in practice, the incident beam contains multiple tagged photons, any of which might provide the requisite energy needed to kinematically generate a given set of final-state tracks. Additionally, there could be multiple ways to recombine or identify final-state particles to arrive at the same event topology. The GlueX reconstruction process considers all valid combinations of tagged photons and sets of reconstructed final-state particles and saves each one in a sub-event called a ``combo''. Each event has at least one of these combos, and some events may have many due to there being many candidate beam photons with similar energies. Since only one combo can be the true event, we need to flatten the data in a way that avoids double counting events. The method for doing so is discussed in \Cref{sub:accidental-subtraction}.

\section{Data Selection for the $K_SK_S$ Channel}\label{sec:data-selection}
These reconstruction steps all take place before we interact with the data. The cut values in \Cref{tab:pid-cuts} are loose, and there is surely background remaining. To remove it, we first must know the potential sources of backgrounds in the $K_S^0K_S^0$ channel, which we will do by simulating a large set of events with many different reaction topologies, passing them through reconstruction, and then comparing the distributions of each topology which makes it through. The simulation uses the \texttt{bggen} Monte Carlo generator, which in turn uses \texttt{PYTHIA}~\cite{Bierlich2022} for event generation and a GlueX implementation of \texttt{Geant4}~\cite{Allison2006,Allison2016,Agostinelli2003}. The relative proportions of reaction topologies do not necessarily reflect the production cross sections we expect to see in data (for instance, there are no resonances included in the $K_S^0K_S^0$ channel), but they should give us a good idea of the kinds of topologies which leak into this channel.

\begin{figure}
  \begin{center}
    \includegraphics[width=0.8\textwidth]{figures/bggen_chisqdof.png}
  \end{center}
  \caption{The reduced $\chi^2$ statistic of the GlueX kinematic fit for the \texttt{bggen} simulation. The stacked histogram contains the signal channel, $K_S^0K_S^0$ as well as the five most prominent background components.}\label{fig:bggen-chisqdof}
\end{figure}

We will first look at the performance of the GlueX kinematic fit on \texttt{bggen} data in \Cref{fig:bggen-chisqdof}. In this plot, we see the signal, which predictably peaks at $\chi^2_\nu \sim 1$, as well as several background topologies. These topologies are labeled by their final state along with any intermediate decaying resonances in square brackets. The top five potential backgrounds, in order of the size of their contribution after reconstruction, are as follows. First, $2\pi^+2\pi^- p$ shares the final state of the signal channel but does not contain any $K_S^0$ intermediate decays. Second, $2\gamma 2\pi^+ 2\pi^- p [\pi^0, \omega]$ contains an $\omega$ which decays to $\pi^+\pi^-\pi^0$, followed by the $\pi^0$ decaying to $2\gamma$. The addition of extra photons is common in these backgrounds, since missing the photon detections makes the final state identical to the signal. The next two reactions, $2\gamma 2\pi^+ 2\pi^- p [\pi^0]$ and $4\gamma 2\pi^+ 2\pi^- p [2\pi^0, \omega]$ both include an undetected $\pi^0$. Finally, $2\pi^+\pi^- K^- p[K_S^0]$ contains a $K^-$ which is mistakenly identified as a $\pi^-$.

It seems sensible to make a cut on the $\chi^2_\nu$ of the kinematic fit, since the signal is clearly favored at lower values. Below $\chi^2_\nu < 10$, the background topologies are dominated by the $4\pi$ channel which does not contain kaons. As such, this source of background will be our primary focus for removal.

\subsection{Fiducial Cuts}\label{sub:fiducial-cuts}
To get a clearer picture of the data, we will first make a loose cut selecting only events with $\chi^2_\nu < 10$. We can then generate phase-space Monte Carlo for the $4\pi$ background reaction and search for potential ways to separate it from the signal by comparing the distributions of common kinematic variables.

\begin{figure}
  \begin{center}
    \includegraphics[width=0.8\textwidth]{ext/analysis/plots/chisqdof_combined_None_None_None.png}
  \end{center}
  \caption{The normalized distributions of $\chi^2_\nu$ for the true data and phase-space signal and $4\pi$ background Monte Carlo.}\label{fig:data-combined-chisqdof}
\end{figure}

In \Cref{fig:data-combined-chisqdof}, we can see the various distributions of the data and each simulated channel. Unfortunately, the data distribution resembles the $4\pi$ background distribution much better than the signal Monte Carlo, but now we also know that the signal decays rapidly at higher values of $\chi^2_\nu$, so a tighter selection might improve the signal-to-background ratio, even if we cannot immediately see a peak in the data near unity. It is difficult to choose a value at which to cut in this variable, but the choice is largely arbitrary. We will be using a statistical weighting method in \Cref{sec:splot} to subtract the non-strange background represented by the $4\pi$ Monte Carlo, but we must make some choice on $\chi^2_\nu$ before that step in the analysis to reduce other backgrounds mentioned in \Cref{sec:data-selection}. For simplicity, we will conduct the entire analysis using a selection of $\chi^2_\nu < 3.0$, and we will later select several other values on either side of this cut to study its systematic effect in \Cref{sec:systematic-studies}.

As mentioned in \Cref{sec:data-selection}, the asymmetry in missing energy in \Cref{fig:me-combined} likely comes from background channels with photons which were missed in reconstruction. The selection on $\chi^2_\nu$ greatly reduces the contributions from these backgrounds, as can be seen in \Cref{fig:me-combined-chisqdof-3.0}

\begin{figure}
  \begin{center}
    \includegraphics[width=0.8\textwidth]{ext/analysis/plots/me_combined_chisqdof_3.0_None_None_None.png}
  \end{center}
  \caption{Normalized distributions of the missing energy ($E_i - E_f$) for data, phase-space signal Monte Carlo, and $4\pi$ background Monte Carlo after a selection of $\chi^2_\nu < 3.0$ is applied.}\label{fig:me-combined-chisqdof-3.0}
\end{figure}


\begin{figure}
  \begin{center}
    \includegraphics[width=0.8\textwidth]{ext/analysis/plots/mm2_combined_chisqdof_3.0_None_None_None.png}
  \end{center}
  \caption{The normalized distributions of missing mass squared for the true data and phase-space signal and $4\pi$ background Monte Carlo after a selection of $\chi^2_\nu < 3.0$ is applied.}\label{fig:mm2-combined-chisqdof-3.0}
\end{figure}

Another common selection which can be made is on the squared missing mass of an event, which is just the square of the difference between the initial- and final-state four-momenta. This variable, visualized in \Cref{fig:mm2-combined-chisqdof-3.0}, predictably peaks at zero for the data, signal Monte Carlo, and background Monte Carlo, and the overall distributions for the signal and backgound simulated events are nearly identical. This indicates that a selection on this variable would not be very useful at distinguishing the signal from the background.

There are two more minor selections which we will perform on the data. The first is a selection on the invariant mass of $K_S^0K_S^0$ at $\text{IM} < \SI{2.0}{\giga\electronvolt}$. This is done because models we use later to describe the data do not include any higher-mass resonances, and data past this point causes issues with the mass-dependent fits in \Cref{sec:mass-dependent-fits}. The other is a selection on the $z$-vertex of the target proton. We know the exact location of the target with respect to the detector, and we only want to deal with events which we know originated inside this target. The distribution of the $z$-vertex values can be seen in \Cref{fig:protonz-combined}. We will select events with $\SI{50}{\centi\meter} < z < \SI{80}{\centi\meter}$, as this is the known length and position of the target relative to the detector elements. While there is an indication of events in the signal Monte Carlo in the regions which are removed, we know that while these events originated from the target, their $z$-vertex has been misidentified, so we should remove them anyway.


\begin{figure}
  \begin{center}
    \includegraphics[width=0.8\textwidth]{ext/analysis/plots/protonz_combined_None_None_None.png}
  \end{center}
  \caption{The normalized distributions of the target proton $z$-vertex for the true data and phase-space signal and $4\pi$ background Monte Carlo (no cuts applied).}\label{fig:protonz-combined}
\end{figure}

An impetus for studying this channel was originally a search for excited hyperons\textemdash baryons with strange quark content. In particular, $\Sigma^+$ resonances should form in the invariant mass distribution of $K_S^0 p$, where the other kaon is required to conserve strangeness in the strong production of this baryon. There is a symmetry between the identical kaons which must first be accounted for. We could pair either kaon with the proton to form the $\Sigma^+$, but there is a more likely pairing which can be found by first recognizing that the heavier hyperon will generally move more backwards in the center-of-momentum frame than the lighter bachelor $K_S^0$. Therefore, the kaon which is moving in a further backward direction is more likely to correspond with the hyperon decay candidate. We can sort the kaons by their $\cos\theta$ in the center-of-momentum frame and call the more backward-going kaon $K_{S,B}^0$. The invariant mass spectrum of $K_{S,B}^0 p$ can be seen in \Cref{fig:baryon-mass-data-pz-chisqdof-3.0}. The enhancement around $\SI{1.7}{\giga\electronvolt}$ likely corresponds to some combination of the $\Sigma^+(1660)$, $\Sigma^+(1670)$, $\Sigma^+(1750)$, and $\Sigma^+(1775)$ resonances. We do not see much indication of states above $\SI{2}{\giga\electronvolt}$.

\begin{figure}
  \begin{center}
    \includegraphics[width=0.8\textwidth]{ext/analysis/plots/baryon_mass_data_pz_chisqdof_3.0_None_None_None.png}
  \end{center}
  \caption{The mass distribution of the backward-going $K_S^0$ combined with the proton. Only the proton $z$-vertex and $\chi^2_\nu$ cuts are applied, as the cut on invariant mass reduces much of the visible baryon contribution which tends to show up at higher $K_S^0K_S^0$ invariant mass.}\label{fig:baryon-mass-data-pz-chisqdof-3.0}
\end{figure}


This angle also gives us a handle on separating baryonic and mesonic topologies. If we plot the $\cos\theta$ of $K_{S,B}^0$ against the invariant mass of $K_{S,B}^0 p$, as in \Cref{fig:ksb-costheta-v-baryon-mass-data-pz-chisqdof-3.0}, we find a strong and somewhat expected dependence. The majority of the baryonic contribution occurs at angles with $\cos\theta < 0$. Likewise, the mesonic contribution is mostly confined to angles of $\cos\theta > 0$, as seen in figures \Cref{fig:ksb-costheta-v-meson-mass-data-pz-chisqdof-3.0,fig:meson-mass-data-pz-masscut-chisqdof-3.0-mesons}. While it may not be obvious that there even are meson resonances here, we will see their contributions much more clearly after the statistical weighting procedures defined in \Cref{sec:splot} have been applied, after which we will revisit these plots. We will also conduct our analysis without this selection in \Cref{sec:systematic-studies} for completeness. We can also isolate the baryon contributions by reversing this selection, and the result of this is seen in \Cref{fig:baryon-mass-data-pz-masscut-chisqdof-3.0-baryons}.

\begin{figure}
  \begin{center}
    \includegraphics[width=0.8\textwidth]{ext/analysis/plots/ksb_costheta_v_baryon_mass_data_pz_chisqdof_3.0_None_None_None.png}
  \end{center}
  \caption{The center-of-momentum azimuthal angle of $K_{S,B}^0$ plotted against the invariant mass of $K_{S,B}^0 p$. The baryonic contribution occurs mostly at far backward angles, while mesons appear at the far forward angles. Only the proton $z$-vertex and $\chi^2_\nu$ cuts are applied, as the cut on invariant mass reduces much of the visible baryon contribution which tends to show up at higher $K_S^0K_S^0$ invariant mass.}\label{fig:ksb-costheta-v-baryon-mass-data-pz-chisqdof-3.0}
\end{figure}

\begin{figure}
  \begin{center}
    \includegraphics[width=0.8\textwidth]{ext/analysis/plots/ksb_costheta_v_meson_mass_data_pz_chisqdof_3.0_None_None_None.png}
  \end{center}
  \caption{The center-of-momentum azimuthal angle of $K_{S,B}^0$ plotted against the invariant mass of $K_S^0K_S^0$. The baryonic contribution occurs mostly at far backward angles, while mesons appear at the far forward angles. Only the proton $z$-vertex and $\chi^2_\nu$ cuts are applied, as the cut on invariant mass reduces much of the visible baryon contribution which tends to show up at higher $K_S^0K_S^0$ invariant mass.}\label{fig:ksb-costheta-v-meson-mass-data-pz-chisqdof-3.0}
\end{figure}

\begin{figure}
  \begin{center}
    \includegraphics[width=0.8\textwidth]{ext/analysis/plots/meson_mass_data_pz_masscut_chisqdof_3.0_mesons_None_None.png}
  \end{center}
  \caption{The invariant mass of $K_S^0K_S^0$ across all datasets after all fiducial selections are applied.}\label{fig:meson-mass-data-pz-masscut-chisqdof-3.0-mesons}
\end{figure}

\begin{figure}
  \begin{center}
    \includegraphics[width=0.8\textwidth]{ext/analysis/plots/baryon_mass_data_pz_masscut_chisqdof_3.0_baryons_None_None.png}
  \end{center}
  \caption{The invariant mass of $K_S^0K_S^0$ across all datasets after all fiducial selections are applied, but with the baryon rejection cut reversed to select baryons.}\label{fig:baryon-mass-data-pz-masscut-chisqdof-3.0-baryons}
\end{figure}

Finally, we can examine some of the typical detector plots which are used to separate and identify charged particles, namely the energy loss from these charged tracks as measured by the FDC and CDC. As we can see in \Cref{fig:dedx-v-p-cdc-proton-pz-masscut-chisqdof-3.0-mesons,fig:dedx-v-p-cdc-pions-pz-masscut-chisqdof-3.0-mesons}, the data recorded in the CDC after the previous fiducial selections is in agreement with that of the signal Monte Carlo. Next, in the FDC, we do not retain a significant number of proton tracks due to the baryon rejection cut, but enough pions are recorded in the FDC to reveal a discrepancy between data and Monte Carlo, as seen in \Cref{fig:dedx-v-p-fdc-pions-pz-masscut-chisqdof-3.0-mesons}. The signal Monte Carlo here tells us that out of the two horizontal bands, the upper band coincides with real pions, while the other band is likely some background. While the total number of pions which end up in the FDC is small, it is still important to remove these background events, and we will do so with a selection on the energy loss {\color{red}Note to committee, this is a work-in-progress which we do not expect will significantly effect the rest of the thesis}.


\begin{figure}
    \centering
    \begin{subfigure}{0.45\textwidth}
        \includegraphics[width=\linewidth]{ext/analysis/plots/dedx_v_p_cdc_proton_sigmc_pz_masscut_chisqdof_3.0_mesons_None_None.png}
        \caption{Signal Monte Carlo}
    \end{subfigure}
    \hfill
    \begin{subfigure}{0.45\textwidth}
        \includegraphics[width=\linewidth]{ext/analysis/plots/dedx_v_p_cdc_proton_data_pz_masscut_chisqdof_3.0_mesons_None_None.png}
        \caption{Data}
    \end{subfigure}
    \caption{The energy loss in the CDC for protons after all fiducial cuts are applied.}\label{fig:dedx-v-p-cdc-proton-pz-masscut-chisqdof-3.0-mesons}
\end{figure}

\begin{figure}
    \centering
    \begin{subfigure}{0.45\textwidth}
        \includegraphics[width=\linewidth]{ext/analysis/plots/dedx_v_p_cdc_piplus1_sigmc_pz_masscut_chisqdof_3.0_mesons_None_None.png}
        \caption{Signal Monte Carlo}
    \end{subfigure}
    \hfill
    \begin{subfigure}{0.45\textwidth}
        \includegraphics[width=\linewidth]{ext/analysis/plots/dedx_v_p_cdc_piplus1_data_pz_masscut_chisqdof_3.0_mesons_None_None.png}
        \caption{Data}
    \end{subfigure}
    \vspace{1em}
    \begin{subfigure}{0.45\textwidth}
        \includegraphics[width=\linewidth]{ext/analysis/plots/dedx_v_p_cdc_piminus1_sigmc_pz_masscut_chisqdof_3.0_mesons_None_None.png}
        \caption{Signal Monte Carlo}
    \end{subfigure}
    \hfill
    \begin{subfigure}{0.45\textwidth}
        \includegraphics[width=\linewidth]{ext/analysis/plots/dedx_v_p_cdc_piminus1_data_pz_masscut_chisqdof_3.0_mesons_None_None.png}
        \caption{Data}
    \end{subfigure}
    \caption{The energy loss in the CDC for pions after all fiducial cuts are applied.}\label{fig:dedx-v-p-cdc-pions-pz-masscut-chisqdof-3.0-mesons}
\end{figure}

\begin{figure}
    \centering
    \begin{subfigure}{0.45\textwidth}
        \includegraphics[width=\linewidth]{ext/analysis/plots/dedx_v_p_fdc_piplus1_sigmc_pz_masscut_chisqdof_3.0_mesons_None_None.png}
        \caption{Signal Monte Carlo}
    \end{subfigure}
    \hfill
    \begin{subfigure}{0.45\textwidth}
        \includegraphics[width=\linewidth]{ext/analysis/plots/dedx_v_p_fdc_piplus1_data_pz_masscut_chisqdof_3.0_mesons_None_None.png}
        \caption{Data}
    \end{subfigure}
    \vspace{1em}
    \begin{subfigure}{0.45\textwidth}
        \includegraphics[width=\linewidth]{ext/analysis/plots/dedx_v_p_fdc_piminus1_sigmc_pz_masscut_chisqdof_3.0_mesons_None_None.png}
        \caption{Signal Monte Carlo}
    \end{subfigure}
    \hfill
    \begin{subfigure}{0.45\textwidth}
        \includegraphics[width=\linewidth]{ext/analysis/plots/dedx_v_p_fdc_piminus1_data_pz_masscut_chisqdof_3.0_mesons_None_None.png}
        \caption{Data}
    \end{subfigure}
    \caption{The energy loss in the FDC for pions after all fiducial cuts are applied. {\color{red}Note to committee: The extra structure here will be addressed with a cut, but it will have minimial impact on this thesis and take several days to reprocess data. Expect an update to this section.}}\label{fig:dedx-v-p-fdc-pions-pz-masscut-chisqdof-3.0-mesons}
\end{figure}

In total, we have the fiducial selections on the data illustrated in \Cref{tab:fiducial-cuts}.

\begin{table}
  \begin{center}
    \begin{tabular}{cc}\toprule
      Variable & Selected Values \\\midrule
      $\chi^2_\nu$ & $\chi^2_\nu < 3.0$ \\
      Mass of $K_S^0K_S^0$ & $ m < \SI{2.0}{\giga\electronvolt} $ \\
      Target proton-$z$ & $\SI{50}{\centi\meter} < z < \SI{80}{\centi\meter}$ \\
      $\cos\theta_{\text{CM}}$ of $K_{S,B}^0$ & $ \cos\theta_{\text{CM}} > 0.0 $ \\\bottomrule
    \end{tabular}
    \caption{Fiducial cuts performed after event reconstruction.}\label{tab:fiducial-cuts}
  \end{center}
\end{table}

\subsection{Accidental Subtraction}\label{sub:accidental-subtraction}
Before we can discuss subtracting the $4\pi$ background, we need to deal with another source of background which we cannot avoid with cuts. As mentioned in \Cref{subsub:combos}, each event in the dataset holds multiple ``combos''\textemdash alternate hypotheses for the set or ordering of the particles which make up the topology. For example, in the $K_S^0K_S^0$ final state, there are two identical $\pi^+$ and $\pi^-$ particles. The choice of which $\pi^+$ goes with which $\pi^-$ to form each kaon is partially constrained by the kinematic fit, but in the case where both pion combinations yield results which pass through the entire reconstruction and particle identification process, both are included as separate combos in the same event. The fiducial selections we performed in \Cref{sub:fiducial-cuts} eliminate many of these combos, since the incorrect combinations will sometimes have very high $\chi^2_\nu$, but there may be some remaining at the end of these selections. For this combinatoric case, we can reduce the effect of incorrect combinations by only selecting the combo with the lowest $\chi^2_\nu$ in each event. While this does not guarantee that we will have the correct combination, it will prevent us from double-counting, and it is more likely that the best $\chi^2_\nu$ is the true event.

Additionally, we can get combos from different beam photon combinations. If multiple tagged photons have energies which are compatible with the reaction, each is included as a separate combo. It can also be the case that the true photon was simply not reconstructed, and some incorrect photon which was close enough was matched with the combo instead. In either case, we have multiple combos which do not correspond to true events. We call both of these cases ``accidental'' combos. To account for this, we first recall that the accelerator produces beam bunches every $\SI{4}{\nano\second}$ in radio-frequency (RF) bunches. A beam bunch consistent with a given reconstructed event is labeled ``in-time'', while the surrounding bunches are called ``out-of-time''. We can reduce the impact of these accidental combos by estimating the contribution they have on the in-time events and using out-of-time events as a negatively-weighted background estimation. Since the in-time accidentals are indistinguishable from true events, we must use other types of events which also pass through kinematic selections in our subtraction, and the out-of-time events can be used for this purpose. The estimation of the scaling for these out-of-time contributions has been calculated via systematic studies of the TAGM/TAGH components. For our data, we choose to include eight bunches of accidentals, although we cut out the bunches neighboring the in-time peak to avoid including in-time events in our subtraction. The distribution of the difference in RF times with respect to the event time is shown in \Cref{fig:rf-unweighted-data-pz-masscut-chisqdof-3.0-mesons}. In this figure, the in-time events are located in the peak centered at zero, and we cut the peaks at $\pm \SI{4}{\nano\second}$ on either side\footnote{The exact value used is $\SI{4.008016032}{\nano\second}$.}. The remaining three out-of-time peaks on either side are given a negative weight (approximately $1/6$th for each) while the in-time peak gets a weight of $1$.

To unify this accidental subtraction with the combinatoric reduction done by selecting the best $\chi^2_\nu$, we first perform the $\chi^2_\nu$ selection, which reduces each event to a single combo (some of which might be in the out-of-time peaks), and then we carry out the accidental subtraction. At this stage, the only background which remains unaccounted for is that of incorrect topologies disguised as our signal.

\begin{figure}
  \begin{center}
    \includegraphics[width=0.8\textwidth]{ext/analysis/plots/rf_unweighted_data_pz_masscut_chisqdof_3.0_mesons_None_None.png}
  \end{center}
  \caption{The time difference between the RF time recorded by the tagger and the event time. The main peak in the middle contains in-time events while the three peaks on either side are the out-of-time events added to the dataset to simulate accidental combos in the main peak.}\label{fig:rf-unweighted-data-pz-masscut-chisqdof-3.0-mesons}
\end{figure}

\section{sPlot Weighting}\label{sec:splot}
At this stage in the analysis, we have no more simple cuts which can improve the signal-to-background ratio in the dataset, but we know there must still be background remaining, as is indicated by the excess events with small kaon rest-frame lifetimes seen in \Cref{fig:rfl-pre-splot}. In this figure, we see that one of the intrinsic properties of a $K_S^0$, its well-known lifetime, is not distinct in the data. Rather, we seem to have at least two exponential slopes in the rest-frame lifetime distribution of each kaon, one which is close to what we see in signal Monte Carlo, and another which is similar to the $4\pi$ background Monte Carlo. We must now turn to more elaborate methods of separating the signal from this potential background seepage. The primary method we will use to do this is sPlot\cite{pivk_splot_2005}\footnote{This is stylized as ${}_s\mathcal{P}lot$ in the original paper, but I find this tedious to type and to read.}, a weighting scheme which corrects the na\"ive probabilistic weights one might first think to construct (dubbed ``inPlot''). We begin by giving a basic explanation of inPlot before describing the sPlot correction.

% TODO: these figures need sizing adjustments (to text as well)
\begin{figure}
  \begin{center}
    \begin{subfigure}[b]{.7\columnwidth}
      \includegraphics[width=1\linewidth]{rfl_data_chisqdof_3.0.png}
      \caption{}
      \label{fig:rfl-data}
    \end{subfigure}
    \begin{subfigure}[b]{.7\columnwidth}
      \includegraphics[width=1\linewidth]{rfl_accmc_chisqdof_3.0.png}
      \caption{}
      \label{fig:rfl-accmc}
    \end{subfigure}
    \begin{subfigure}[b]{.7\columnwidth}
      \includegraphics[width=1\linewidth]{rfl_bkgmc_chisqdof_3.0.png}
      \caption{}
      \label{fig:rfl-bkgmc}
    \end{subfigure}
  \end{center}
  \caption{The rest-frame lifetime of kaons in (a) data, (b) signal Monte Carlo, and (c) background Monte Carlo. The data distribution clearly contains two exponential slopes: a peak which resembles the $4\pi$ Monte Carlo distribution, and a tail of true $K_S^0$s which resembles the signal Monte Carlo.}\label{fig:rfl-pre-splot}
\end{figure}

For all of the statistical weighting methods which will be mentioned here, we need some model for the signal and background probability distribution functions (PDFs) for some ``discriminating'' variable. This variable is called ``discriminating'' because it the variable for which we know the shape of these distributions beforehand. The usual example is a ``bump-on-a-background'', in which the discriminating variable may be a mass distribution ($m$) where signal events show up as a peaking structure while background events are more uniformly distributed. In such situations, it is common to use the extremes of the mass distribution (sidebands) as estimates of the background everywhere, weighting these events negatively while the events in the peak are weighted positively (a sideband subtraction). Rather than specifying peak and sideband regions, we can fit the mass distribution to some mixture of a signal (peak) PDF $f_S(m)$ and a background (flat) PDF $f_B(m)$. From such a fit, we obtain estimated number of signal ($N_S$) and background ($N_B$) events in our dataset (and possibly some shape parameters for the signal and background PDFs). We could then assign weights to each event as in \Cref{eq:inplot-weights-mass},

\begin{equation}
  w(m) = \frac{N_S f_S(m)}{N_S f_S(m) + N_B f_B(m)}
  \label{eq:inplot-weights-mass}
\end{equation}

We might want to look at the ``signal'' inPlots for the decay angles $\theta$ and $\varphi$ (control variables) in the helicity system after calculating the inPlot weights from a fit to the mass distribution (discriminating variable). However, as shown by Pivk and Le Diberder\cite{pivk_splot_2005}, we can only use inPlot in cases where the control variables are totally correlated with the discriminating variable\footnote{In practice, more than one discriminating variable can be used.} $y$. In other words, our example would only be valid if $\theta = \theta(m)$ and $\phi = \phi(m)$. For the time being, let us assume that this is not the case, and that we wish to use the distribution of some variable which is totally uncorrelated with the variables we are plotting and analyzing\footnote{Total (un)correlation is a very strict requirement, but we will later see that small modifications to the sPlot method can permit amounts of correlation between the two extremes.}. A correction term can be applied to give us the sPlot version of \Cref{eq:inplot-weights-mass},

\begin{equation}
  w(y) = \tilde{w}(y)\frac{V_{SS}f_S(y) + V_{SB}f_B(y)}{N_S f_S(y) + N_B f_B(y)},\quad \text{where } V^{-1}_{ij} = \sum_{y} \frac{\tilde{w}(y)f_i(y)f_j(y)}{\left(N_S f_S(y) + N_B f_B(y)\right)^2}
  \label{eq:splot-weights}
\end{equation}

where $y$ represents any set of discriminating variables (not necessarily a mass), and $\tilde{w}(y)$ is any pre-existing weight associated with the event (weights from accidental subtraction, for instance). The $V^{-1}$ matrix can also be understood as the covariance matrix between the free parameters $N_S$ and $N_B$ in the fit of the signal-background mixture, $V^{-1}_{ij} = -N\pdv[2]{\ln\mathcal{L}}{N_i}{N_j}$, although there is reason to believe that direct calculation by inverting the Hessian matrix from the fit will lead to less accurate results than the manual calculation method given in \Cref{eq:splot-weights}\cite{dembinski_custom_2022}.

Now that we have a method of assigning weights, we must pick the discriminating variables. As mentioned, these weighting methods work well on the classic ``bump-on-a-background'' distributions because it is easy to identify the signal and background PDFs, but because the mass of the kaons is constrained in the kinematic fit, the fitted mass of each kaon is just a $\delta$-function and combination of measured masses for each $\pi^+\pi^-$ pair will yield a Normal distribution with little to no apparent background (by construction), so we must be a bit more clever in selecting discriminating variables. By examining the BGGEN analysis done in {\color{red}[TODO: PREVIOUS SECTION]}, we can see that most likely sources of background arise when the intermediate kaons are absent from the reaction: $\gamma p \to 4\pi p$. This reaction has the $K_SK_S$ final state, so pairs of pions which reconstruct close enough to kaons will be almost indistinguishable in the data. However, they differ in one key way, namely that the $K_S$ intermediate contains a strange quark while the $\pi^+\pi^-$ decay state does not, so such a decay must occur via the weak interaction, which is notably slower than the strong interaction which would produce pion pairs with no intermediate kaon. In other words, while the signal's rest-frame lifetime distribution should have an exponential slope near the $K_S$ lifetime, the background would theoretically have nearly zero rest-frame lifetime for every event, or a much smaller exponential slope in practice\footnote{An exponential distribution is just what best fits the rest-frame lifetime distribution in the $4\pi$ Monte Carlo and has no physical implication.}.

Therefore, we will begin by generating both a signal and background dataset in Monte Carlo. We then interpret both datasets as if they were our desired channel by running them through the GlueX reconstruction and reaction filter, as well as all of our selections up to this point. We can then fit the rest-frame lifetime of each dataset to an exponential model,

\begin{equation}
  f(t; \lambda) = \lambda \exp{-\lambda t},
  \label{eq:splot-exponential}
\end{equation}
where $\lambda \equiv 1/\tau$, the lifetime of the kaon in question. Since we have two independently decaying kaons, we should really form a joint distribution for both, where we will assume each kaon has the same average lifetime:
\begin{equation}
  f(t_1, t_2; \lambda) = \lambda^2 \exp{-\lambda t_1}\exp{-\lambda t_2}
  \label{eq:splot-exponential_joint}
\end{equation}
Both the signal and background distributions can be modeled in this way, giving us only two free parameters, $\lambda_S$ and $\lambda_B$ for the signal and background respectively, to fit.

We can then use a mixture of exponential distributions with both signal and background slopes to fit the entire dataset:
\begin{equation}
  g(t_1, t_2; z, \lambda_S, \lambda_B) \equiv z f(t_1, t_2; \lambda_S) + (1-z) f(t_1, t_2; \lambda_B)
  \label{eq:splot-mixture-exp}
\end{equation}

where $z$ is the signal fraction of the total number of events $N$. From its fit value, we can determine values of $N_S = z\cdot N$ and $N_B = (1-z)\cdot N$ to use in \Cref{eq:splot-weights} and complete the weighting procedure. We can perform this fit by minimizing the negative log-likelihood function,

\begin{equation}
  -2\ln\mathcal{L}(z, \lambda_S, \lambda_B) = -2\sum_i^N \tilde{w}_i \ln g(t_{1,i}, t_{2,i}; z, \lambda_S, \lambda_B)
  \label{eq:splot-nll-exp}
\end{equation}

where again, we include any pre-existing weights $\tilde{w}$ in the fit.

Examining \Cref{fig:rfl-accmc}, we can see that the expected distribution from signal Monte Carlo is not quite exponential, while the expected contribution from background Monte Carlo in \Cref{fig:rfl-bkgmc} is, at least in the important region of low rest-frame lifetimes. Rather than model the signal distribution analytically, we will instead use the distribution from the Monte Carlo itself as the model by binning the simulated data (a bin width of $\SI{1}{\pico\second}$ seems to give a distribution that is decently smooth in practice). In the following discussions, we will use this binned distribution as the model for the signal component and the exponential distribution in \Cref{eq:splot-exponential_joint} for the background component. Because of this, the signal distribution does not have a slope parameter $\lambda_S$, so the true mixture equation looks like,

\begin{equation}
  g(t_1, t_2; z, \lambda_B) \equiv z \tilde{f}(t_1, t_2) + (1-z) f(t_1, t_2; \lambda_B)
  \label{eq:splot-mixture}
\end{equation}

where $\tilde{f}$ represents the binned distribution, and the equation for the likelihood is,

\begin{equation}
  -2\ln\mathcal{L}(z,  \lambda_B) = -2\sum_i^N \tilde{w}_i \ln g(t_{1,i}, t_{2,i}; z, \lambda_B)
  \label{eq:splot-nll}
\end{equation}

\subsection{Non-Factorizing sPlot}\label{sec:non-factorizing-splot}

Over the course of the previous discussion, it was assumed that the discriminating variables, $t_1$ and $t_2$, were statistically independent from the control variables we wish to use in later analyses. The set of control variables must include all variables we use as inputs to the partial-wave analysis in \Cref{ch:partial-wave-analysis}, including the invariant mass $m$ of the $K_S^0K_S^0$ system and the helicity angles $\theta$ and $\varphi$ of the decay. We should now confirm that the rest-frame lifetimes are totally uncorrelated with these control variables (in other words, show that they are statistically independent). To test for statistical independence between $t_{1,2}$ and a given control variable, we first split our dataset into $M$ evenly-spaced quantiles in that control variable, which ensures each bin gets roughly the same number of events. Next, we calculate the likelihood of a null hypothesis which assumes the variables are statistically independent by fitting all datasets simultaneously with a shared $\lambda_B$ parameter. We then calculate the likelihood of an alternative hypothesis, which assumes statistical dependence, by finding the joint likelihood of independent fits of $\lambda_B$ over each quantile. The result of these fits can be formulated as a likelihood ratio,

\begin{equation}
  \Lambda = -2\ln\frac{\sup \mathcal{L}_{H_0}}{\sup \mathcal{L}_{H_1}} = -2\ln\frac{\sup \prod_i^M \mathcal{L}_i(z_i, \lambda_B)}{\sup \prod_i^M \mathcal{L}_i(z_i, \lambda_{B,i})}
  \label{eq:independence-test}
\end{equation}

where $\mathcal{L}_{H_0}$ and $\mathcal{L}_{H_1}$ are the likelihoods of the null and alternative hypotheses respectively, the supremum indicates we are maximizing these likelihoods (in a maximum likelihood fit), the product $\prod_i^M$ iterates over each quantile of data in the given control variable, and $\mathcal{L}_i$ is the likelihood evaluated over data in the $i$th quantile. $\Lambda$ is $\chi^2$ distributed with $M - 1$ degrees of freedom (the difference between $M + 1$ free parameters in the null hypothesis, a signal fraction $z_i$ for each quantile plus the exponential background slope shared across all quantiles, and $2M$ in the alternative hypothesis, a signal fraction and one exponential slope for each quantile). The factor of $2$ is required because $\ln\mathcal{L}(\theta_1,...,\theta_i) \sim -\frac{1}{2}\chi^2_i$ asymptotically with sample size, according to Wilks' theorem. We can obtain a $p$-value representing the likelihood of the null hypothesis being true by evaluating the $p$-value:

\begin{equation}
  p = 1 - F_{\chi^2_{M-1}}(\Lambda)
  \label{eq:significance-test}
\end{equation}

where $F_{\chi^2_{M-1}}(\Lambda)$ is the cumulative distribution function of a $\chi^2$ distribution with $M-1$ degrees of freedom. Following this procedure for the invariant mass of $K_S^0K_S^0$\footnote{No significant statistical dependence was found for the helicity angles.}, the $p$-values for each number of quantiles can be seen in the first row of \Cref{tab:factorization-results} and in \Cref{fig:data-factorization-fit}, which depicts the case with four quantiles. The calculated $p$-values tend to be very small, implying that we should reject the null hypothesis and accept that the discriminating (rest-frame lifetime) and control (invariant mass of $K_S^0K_S^0$) variables are not statistically independent. This means we cannot use a traditional sPlot to weight our data.

Fortunately, the process for obtaining the correct weights is straightforward, we simply allow for more than one signal and background component in the fit and sum over all signal components when we calculate the final weight values~\cite{dembinski_custom_2022}. Since the weights corresponding to each signal component in the sPlot can be added to each other to obtain a joint weight~\cite{pivk_splot_2005}, \Cref{eq:splot-weights} can be extended to allow multiple signal and background components:

\begin{equation}
  w(x) = \frac{\sum_{j} V_{S_0 j}f_j(x)}{\sum_{k}N_kf_k(x)} + ... + \frac{\sum_{j} V_{S_n j}f_j(x)}{\sum_{k}N_kf_k(x)},\quad \text{where } V_{ij}^{-1} = \sum_{x} \frac{f_i(x)f_j(x)}{\left(\sum_{k} N_kf_k(x)\right)^2}
  \label{eq:splot-weights-factorizing}
\end{equation}

and $S_i$ are the indices of the signal components.

We can further verify the need for non-factorizing sPlot by performing the factorization test described in \Cref{eq:independence-test,eq:significance-test} to the background Monte Carlo.

\begin{equation}
  p = 1 - F_{\chi^2_{M-1}} \quad\text{where }\Lambda = -2\ln\frac{\sup\prod_i^M \mathcal{L}_i(\lambda)}{\sup\prod_i^M \mathcal{L}_i(\lambda_i)}
  \label{eq:independence-test-mc}
\end{equation}

The results of these tests over the background Monte Carlo can also be found in the last two rows of \Cref{tab:factorization-results} and are visualized for four quantiles in \Cref{fig:mc-factorization-fits}. Again, the significantly small $p$-values justify the use of non-factorizing sPlot across the background component, meaning that we need at least two background components in the final sPlot weighting\footnote{We found that a similar analysis of an exponential signal distribution also indicates a significant amount of non-factorization, but the difference in the slopes of each component are too small to give significantly different fits to the true data.}.

\begin{table}
  \begin{center}
    \begin{tabular}{cccc}\toprule
       & \multicolumn{3}{c}{\# quantiles} \\\cmidrule(lr){2-4}
       & 2 & 3 & 4 \\
       Data Type & $p$ & $p$ & $p$ \\\midrule
      Data  & $1.26 \times 10^{-101}$ & $9.94 \times 10^{-103}$ & $4.49 \times 10^{-112}$\\
      $K_S^0K_S^0$ MC  & $<2.23\times 10^{-308}$ & $<2.23\times 10^{-308}$ & $<2.23\times 10^{-308}$\\
      $4\pi$ MC  & $1.87 \times 10^{-05}$ & $1.00 \times 10^{-04}$ & $7.38 \times 10^{-05}$\\\bottomrule
    \end{tabular}
    \caption{The probability of accepting the null hypothesis (that the rest-frame lifetime is statistically independent of the invariant mass of $K_S^0K_S^0$) for the tests described in \Cref{eq:independence-test} for data and \Cref{eq:independence-test-mc} for Monte Carlo with the given number of quantiles. All values are calculated with a $\chi^2_\nu < 3.0$ selection on each type of data over all run period combined. Values listed as $<2.23 \times 10^{-308}$ are nonzero but smaller than the smallest representable 64-bit floating point number}\label{tab:factorization-results}
  \end{center}
\end{table}

\begin{figure}
  \begin{center}
    \includegraphics[width=.8\columnwidth]{factorization_plot_data_chisqdof_3.0_4_quantiles.png}
  \end{center}
  \caption{Exponential slopes from fits over four quantiles in $m(K_S^0K_S^0)$ ($x$-axis) to a mixture of signal (left $y$-axis) and background (right $y$-axis) components. These fits show a definite statistical dependence between rest-frame lifetime and the invariant mass of $K_S^0K_S^0$, as described in \Cref{tab:factorization-results}. All values are calculated with a KinFit $\chi^2_\nu < 3.0$ selection on each type of data over each run period.}\label{fig:data-factorization-fit}
\end{figure}

\begin{figure}
  \begin{center}
    \begin{subfigure}[b]{.8\columnwidth}
      \includegraphics[width=1\linewidth]{factorization_plot_accmc_chisqdof_3.0_4_quantiles.png}
      \caption{}
      \label{fig:mc-factorization-fits-a}
    \end{subfigure}
    \begin{subfigure}[b]{.8\columnwidth}
      \includegraphics[width=1\linewidth]{factorization_plot_bkgmc_chisqdof_3.0_4_quantiles.png}
      \caption{}
      \label{fig:mc-factorization-fits-b}
    \end{subfigure}
  \end{center}
  \caption{Exponential slopes from fits over four quantiles in $m(K_S^0K_S^0)$ ($x$-axis) of the (a) signal and (b) background Monte Carlo rest-frame lifetime distributions. Both show a definite statistical dependence between rest-frame lifetime and the invariant mass of $K_S^0K_S^0$, as described in \Cref{tab:factorization-results}. All values are calculated with a KinFit $\chi^2_\nu < 3.0$ selection on each type of data over each run period.}\label{fig:mc-factorization-fits}
\end{figure}

\subsection{Application of Weights}\label{sec:application-of-weights}

The only thing left to do is determine how many background components we should use in the weighting procedure. To this end, we now turn to the Monte Carlo simulations of the $4\pi$-background. By choosing a number of quantiles in invariant mass corresponding to the number of components, we can fit single exponential distributions to each quantile in the simulated background. For instance, if we chose to use three background components, we would divide the background Monte Carlo into three, and fit each quantile to an exponential distribution to obtain a set of three $\lambda_B$ values. The resulting  $\lambda_B$ values could then be used as a starting point for a multi-component fit to the data. Alternatively, the background slopes could be fixed to the values from the fits to simulations, and only the yields would be allowed to float in the fit to data. We will refer to the first case, where the fit parameters from Monte Carlo are free, as $A$, the case where they are fixed as $B$. To select a model, we can use the relative Akaike Information Criterion (AIC)~\cite{akaike_information_1998} and Bayesian Information Criterion (BIC)~\cite{schwarz_estimating_1978}:
\begin{alignat}{2}
  r\text{AIC} &\equiv \text{AIC} - \text{AIC}_\text{min} \quad\text{where } \text{AIC} &&\equiv 2k - 2\ln\mathcal{L} \\
  r\text{BIC} &\equiv \text{BIC} - \text{BIC}_\text{min} \quad\text{where } \text{BIC} &&\equiv k\ln{N} - 2\ln\mathcal{L} \\
  \label{eq:information-criteria}
\end{alignat}
where $k$ is the number of free parameters and $N$ is the number of events in the dataset. The optimal model will minimize these criteria. In \Cref{tab:splot-model-results}, all of the relative AIC and BIC values are shown. Excluding cases with only one background component (restricting to models which have non-factorizing background components), {\color{red}[TODO]} the minimizing values for most run periods tend to use two or three signal and two background components, and they both use method $B$, where the signal components are fixed to values obtained from Monte Carlo while the background components are initialized at Monte Carlo values but allowed to float in the fit. We will use the minimal non-factorizing model, method $B$ with two signal and two background components, denoted $B(2,2)$, as our weighting method. See \Cref{fig:splot-data-fit} for the result of this fit. The selection of method $B$ is also interesting as it could describe a case where another background which is not modeled in the background Monte Carlo is present and has a similar exponential slope. Since these slopes are free in the fit to the data, they may anticipate this unknown slope better than the fixed case (method $C$), and method $B$ also explicitly assumes the Monte Carlo for true signal kaons is correct and fixes their component slopes (unlike method $A$).

\begin{table}
  \begin{center}
    \begin{tabular}{ccccc}\toprule
    & \multicolumn{2}{c}{\# Components} & \\\cmidrule(lr){2-3}
      Method & Signal & Background & $r\text{AIC}$ & $r\text{BIC}$\\\midrule
      $A$ & $1$ & $1$ & \underline{$0.000$} & \underline{$0.000$}\\
       & $1$ & $2$ & $4.000$ & $25.420$\\
       & $1$ & $3$ & $8.004$ & $50.843$\\
       & $2$ & $1$ & $4.000$ & $25.420$\\
       & $2$ & $2$ & $8.000$ & $50.839$\\
       & $2$ & $3$ & $12.003$ & $76.261$\\
       & $3$ & $1$ & $8.000$ & $50.839$\\
       & $3$ & $2$ & $12.002$ & $76.261$\\
       & $3$ & $3$ & $16.002$ & $101.681$\\\midrule
      $B$ & $1$ & $1$ & $312.365$ & $301.655$\\
       & $1$ & $2$ & $213.928$ & $224.638$\\
       & $1$ & $3$ & $217.324$ & $249.453$\\
       & $2$ & $1$ & $23.929$ & $23.929$\\
       & $2$ & $2$ & $9.758$ & \fcolorbox{red}{white}{$31.177$}\\
       & $2$ & $3$ & $13.629$ & $56.468$\\
       & $3$ & $1$ & $2.001$ & $12.711$\\
       & $3$ & $2$ & \fcolorbox{red}{white}{$6.004$} & $38.134$\\
       & $3$ & $3$ & $10.002$ & $63.551$\\\midrule
      $C$ & $1$ & $1$ & $1695.033$ & $1673.614$\\
       & $1$ & $2$ & $1305.143$ & $1294.433$\\
       & $1$ & $3$ & $1197.528$ & $1197.528$\\
       & $2$ & $1$ & $1661.600$ & $1650.890$\\
       & $2$ & $2$ & $1245.238$ & $1245.238$\\
       & $2$ & $3$ & $1128.344$ & $1139.053$\\
       & $3$ & $1$ & $1661.070$ & $1661.070$\\
       & $3$ & $2$ & $1247.365$ & $1258.074$\\
       & $3$ & $3$ & $1131.169$ & $1152.588$\\\bottomrule
    \end{tabular}
    \caption{Relative AIC and BIC values for each fitting method. The absolute minimum values in each column are underlined, and the minimums excluding models with only one signal or background component are boxed.}\label{tab:splot-model-results}
  \end{center}
\end{table}

\begin{figure}
  \begin{center}
    \includegraphics[width=.8\columnwidth]{splot_fit_data_chisqdof_3.0_splot_B_2s_2b.png}
  \end{center}
  \caption{Fit of \Cref{eq:splot-mixture} to data using method $B(2,2)$. True kaon events are prominent in the tail of the distribution, whereas background events peak strongly near zero. All values are calculated with a KinFit $\chi^2_\nu < 3.0$ selection on each type of data over each run period. The second signal component tends to be very small but non-zero across all datasets.}\label{fig:splot-data-fit}
\end{figure}


\subsection{Neutral Kaon Decay Channels}\label{sub:neutral-kaon-decay-channels}
As previously mentioned, it is possible for a kaon to decay as $K_S^0 \to 2\pi^0 \to 4\gamma$, and in fact we expect this decay to occur about $30\%$ of the time while the $\pi^+\pi^-$ mode covers the remaining $70\%$ of decays\footnote{There are a few other decay modes each with branching fractions of far less than $1\%$.}~\cite{Zyla2020}. It is straightforward to see that the fraction of $K_S^0 K_S^0$ pairs which both decay to charged pions is $70\%\times 70\%=49\%$, so it may seem like ignoring the neutral pion channels leaves over half of the potential data unobserved. While this is true in theory, these channels are much more difficult to reconstruct in practice. First, the acceptance effects on angular distributions will differ due to the photons being detected in the FCAL or BCAL while the charged pions are detected in the CDC or FCD. Second, because both the kaon and $\pi^0$ are neutral, the kaon decay vertex is ``detached'', meaning we cannot reconstruct the decay vertex of $K_S^0 \to 2\pi^0$ nor\footnote{Technically, the decay $\pi^0 \to 2\gamma$ is governed by electromagnetic interactions, so its decay time is much shorter than that of the kaon and is not long enough to be observed by the GlueX detector anyway.} that of $\pi^0 \to 2\gamma$. This makes it impossible to use sPlot on the channel where both kaons decay to neutral pions, since there are no vertices with which to determine the rest-frame lifetime. In the channel with one $K_S^0\to\pi^+\pi^-$ decay, we can perform sPlot using the single rest-frame lifetime distribution, but due to the way the kinematic fit is written, we cannot constrain the mass of a kaon with a detached vertex, meaning we must then perform some background subtraction on the invariant mass distributions of these kaons, introducing a further complication in the analysis. Additionally, these channels have different backgrounds than the one used in this analysis, since the final state can be recombined in more ways than the charged pions. These issues make the total reconstruction efficiency of neutral-pion decay channels lower, since they are less constrained and tend to let in more background than the fully charged-pion decay channel, despite the signal reconstruction efficiency being approximately the same across all channels. While we examined these alternative channels during the analysis, we have not included them in this thesis since the additional work required to refine them is not worth the small gain in statistics, though they may be of interest in future studies when GlueX collects more data.

% GEN => 99,608,746
% KsKs => 4,437,895
% KsPi0Pi0 => 3,819,903
% Pi0Pi0Pi0Pi0 => 799,955
